% Options for packages loaded elsewhere
\PassOptionsToPackage{unicode}{hyperref}
\PassOptionsToPackage{hyphens}{url}
\PassOptionsToPackage{dvipsnames,svgnames,x11names}{xcolor}
%
\documentclass[
  letterpaper,
]{book}

\usepackage{amsmath,amssymb}
\usepackage{iftex}
\ifPDFTeX
  \usepackage[T1]{fontenc}
  \usepackage[utf8]{inputenc}
  \usepackage{textcomp} % provide euro and other symbols
\else % if luatex or xetex
  \usepackage{unicode-math}
  \defaultfontfeatures{Scale=MatchLowercase}
  \defaultfontfeatures[\rmfamily]{Ligatures=TeX,Scale=1}
\fi
\usepackage{lmodern}
\ifPDFTeX\else  
    % xetex/luatex font selection
\fi
% Use upquote if available, for straight quotes in verbatim environments
\IfFileExists{upquote.sty}{\usepackage{upquote}}{}
\IfFileExists{microtype.sty}{% use microtype if available
  \usepackage[]{microtype}
  \UseMicrotypeSet[protrusion]{basicmath} % disable protrusion for tt fonts
}{}
\makeatletter
\@ifundefined{KOMAClassName}{% if non-KOMA class
  \IfFileExists{parskip.sty}{%
    \usepackage{parskip}
  }{% else
    \setlength{\parindent}{0pt}
    \setlength{\parskip}{6pt plus 2pt minus 1pt}}
}{% if KOMA class
  \KOMAoptions{parskip=half}}
\makeatother
\usepackage{xcolor}
\setlength{\emergencystretch}{3em} % prevent overfull lines
\setcounter{secnumdepth}{5}
% Make \paragraph and \subparagraph free-standing
\ifx\paragraph\undefined\else
  \let\oldparagraph\paragraph
  \renewcommand{\paragraph}[1]{\oldparagraph{#1}\mbox{}}
\fi
\ifx\subparagraph\undefined\else
  \let\oldsubparagraph\subparagraph
  \renewcommand{\subparagraph}[1]{\oldsubparagraph{#1}\mbox{}}
\fi


\providecommand{\tightlist}{%
  \setlength{\itemsep}{0pt}\setlength{\parskip}{0pt}}\usepackage{longtable,booktabs,array}
\usepackage{calc} % for calculating minipage widths
% Correct order of tables after \paragraph or \subparagraph
\usepackage{etoolbox}
\makeatletter
\patchcmd\longtable{\par}{\if@noskipsec\mbox{}\fi\par}{}{}
\makeatother
% Allow footnotes in longtable head/foot
\IfFileExists{footnotehyper.sty}{\usepackage{footnotehyper}}{\usepackage{footnote}}
\makesavenoteenv{longtable}
\usepackage{graphicx}
\makeatletter
\def\maxwidth{\ifdim\Gin@nat@width>\linewidth\linewidth\else\Gin@nat@width\fi}
\def\maxheight{\ifdim\Gin@nat@height>\textheight\textheight\else\Gin@nat@height\fi}
\makeatother
% Scale images if necessary, so that they will not overflow the page
% margins by default, and it is still possible to overwrite the defaults
% using explicit options in \includegraphics[width, height, ...]{}
\setkeys{Gin}{width=\maxwidth,height=\maxheight,keepaspectratio}
% Set default figure placement to htbp
\makeatletter
\def\fps@figure{htbp}
\makeatother

\makeatletter
\@ifpackageloaded{bookmark}{}{\usepackage{bookmark}}
\makeatother
\makeatletter
\@ifpackageloaded{caption}{}{\usepackage{caption}}
\AtBeginDocument{%
\ifdefined\contentsname
  \renewcommand*\contentsname{Table of contents}
\else
  \newcommand\contentsname{Table of contents}
\fi
\ifdefined\listfigurename
  \renewcommand*\listfigurename{List of Figures}
\else
  \newcommand\listfigurename{List of Figures}
\fi
\ifdefined\listtablename
  \renewcommand*\listtablename{List of Tables}
\else
  \newcommand\listtablename{List of Tables}
\fi
\ifdefined\figurename
  \renewcommand*\figurename{Figure}
\else
  \newcommand\figurename{Figure}
\fi
\ifdefined\tablename
  \renewcommand*\tablename{Table}
\else
  \newcommand\tablename{Table}
\fi
}
\@ifpackageloaded{float}{}{\usepackage{float}}
\floatstyle{ruled}
\@ifundefined{c@chapter}{\newfloat{codelisting}{h}{lop}}{\newfloat{codelisting}{h}{lop}[chapter]}
\floatname{codelisting}{Listing}
\newcommand*\listoflistings{\listof{codelisting}{List of Listings}}
\makeatother
\makeatletter
\makeatother
\makeatletter
\@ifpackageloaded{caption}{}{\usepackage{caption}}
\@ifpackageloaded{subcaption}{}{\usepackage{subcaption}}
\makeatother
\ifLuaTeX
  \usepackage{selnolig}  % disable illegal ligatures
\fi
\usepackage{bookmark}

\IfFileExists{xurl.sty}{\usepackage{xurl}}{} % add URL line breaks if available
\urlstyle{same} % disable monospaced font for URLs
\hypersetup{
  pdftitle={Notas de Aprendizaje y Comportamiento Adaptable},
  pdfauthor={Arturo Bouzas},
  colorlinks=true,
  linkcolor={Maroon},
  filecolor={Maroon},
  citecolor={Blue},
  urlcolor={Blue},
  pdfcreator={LaTeX via pandoc}}

\title{Notas de Aprendizaje y Comportamiento Adaptable}
\author{Arturo Bouzas}
\date{2025-03-15}

\begin{document}
\frontmatter
\maketitle

\renewcommand*\contentsname{Table of contents}
{
\hypersetup{linkcolor=}
\setcounter{tocdepth}{2}
\tableofcontents
}
\mainmatter
\bookmarksetup{startatroot}

\chapter{Notas de Aprendizaje y Comportamiento
Adaptable}\label{notas-de-aprendizaje-y-comportamiento-adaptable}

\bookmarksetup{startatroot}

\chapter*{Prefacio}\label{prefacio}
\addcontentsline{toc}{chapter}{Prefacio}

\markboth{Prefacio}{Prefacio}

Este es un libro de notas para los cursos de ``Aprendizaje y
Comportamiento Adpatable'', que se imparten el la facultad de Psicología
de la Universidad Nacional Autónoma de México

El proyecto de esta página web fue financiado por el proyecto PAPIME
PE302221

Arturo Bouzas.

\bookmarksetup{startatroot}

\chapter{Introducción}\label{introducciuxf3n}

La forma en que se organizan y presentan..

\bookmarksetup{startatroot}

\chapter{Principios de la Selección
Natural}\label{principios-de-la-selecciuxf3n-natural}

\begin{quote}
``Nothing in Biology (Psychology) Makes Sense Except in the Light of
Evolution''. Dobzhansky.
\end{quote}

Como vimos en una nota anterior, la influencia de Darwin y el
pensamiento seleccionista sobre la Psicología fue enorme. Hoy en día,
para entender el comportamiento adaptado y adaptable, resulta
indispensable conocer los principios más generales de la teoría de la
selección natural. El propósito de esta nota es presentar estos
principios. Para un tratamiento extenso del tema, revise las
recomendaciones al final de la nota.

Para cualquier observador del mundo natural, dos propiedades le
parecerán sorprendentes y merecedoras de contar con una explicación. La
primera de ellas es la enorme variabilidad en morfología, fisiología y
comportamiento de los organismos que habitan este planeta. La segunda,
es que esta variabilidad parece estar finamente ajustada (adaptada) a
las características del entorno que habitan los organismos.

\subsubsection{1. Variabilidad}\label{variabilidad}

Un repaso de nuestra experiencia cotidiana, visitas a zoológicos y
videos de historia natural en YouTube, nos alerta a la enorme variedad
de organismos que pueblan nuestro planeta: desde organismos
unicelulares, hongos, medusas, plantas de múltiples tamaños, insectos,
peces, aves, mamíferos y desde luego humanos. La variabilidad no es solo
a nivel de la morfología, se da también en los mecanismos fisiológicos y
en el comportamiento de los seres vivos. Hay organismos que se
desplazan, otros que no; organismos que se alimentan de un solo producto
y otros que comen de todo; organismos que regulan su temperatura, otros
que no; organismos que se reproducen sexualmente, otros que no;
organismos que tienen una sola cría, otros que depositan cientos de
huevos. Aún entre una clase, como la de los perros, existen subtipos de
todo tamaño, conformación, nivel de actividad, niveles de apego y
agresividad.

Actualmente se estiman entre 10 millones a un billón de especies. Solo
entre mamíferos existen aproximadamente 5,500 especies, de insectos
91,000 y no deja de sorprender las 250,000 especies de escarabajos (buen
tino de los ``Beatles''). Solo en el intestino humano hay 140,000
especies virales, y se estiman entre un millón a 1 billón de especies de
bacterias en nuestro planeta. Bacterias y virus representan la mayoría
de los organismos biológicos.

La variabilidad observada cambia tanto a lo largo del espacio como a lo
largo del tiempo. La variabilidad que registramos varía en sí misma
dependiendo del lugar del planeta donde se lleve a cabo la observación.
Lo que observamos en el desierto de Sonora es muy diferente a lo que
observamos en la selva chiapaneca. Fue esta variabilidad lo primero que
impactó a Darwin en su recorrido en el Beagle.

A mediados del siglo XIX, también se sabía que la variabilidad cambiaba
a lo largo del tiempo. No solo había especies que ya habían
desaparecido, sino, y esto era lo importante, ahora se sabía que también
habían existido especies de las cuales no se tenían registros pero que
habitaron la Tierra en etapas geológicas más tempranas. Era claro que el
mundo biológico no había sido ``creado'' a un mismo tiempo y que podía
hablarse de la evolución de los organismos. Gracias a los avances en la
datación de capas geológicas, el análisis de fósiles nos permite
determinar con mayor precisión la fecha de aparición de las diferentes
especies. Estos datos confirman las predicciones de la teoría de Darwin
sobre la evolución gradual y la ascendencia común.

Ante el panorama descrito, surgen un número de preguntas, ¿por qué no
todos los organismos son iguales?, ¿cuál es el origen de esa gran
variabilidad?, ¿cómo dar cuenta de ella y de su distribución en el
espacio y en el tiempo? Fueron estas preguntas las que dieron origen a
la noción de la evolución.

\subsubsection{2. Adaptación}\label{adaptaciuxf3n}

Una segunda observación que requiere explicación, es el sorprendente
ajuste (adaptación) de las características morfológicas, fisiológicas y
de comportamiento , a las características del entorno donde se
desenvuelve un organismo. No tan solo existe una inmensa variabilidad,
sino que esta está correlacionada con las propiedades de entornos que
varían espacial y temporalmente. El ejemplo clásico de una adaptación
morfológica lo observó Darwin al visitar los archipiélagos de las islas
Galápagos y de Hawaii. Él encontró que la forma del pico de pequeños
pájaros genéticamente relacionados, se ajustaba al tipo de alimento en
la isla que habitaban. Cómo puede verse en la siguiente figura, el pico
podía ser largo y delgado, apropiado para acceder al néctar dentro de
una flor, o corto y fuerte para poder romper y comer semillas.

\subsection{3. Principios de la Selección
Natural}\label{principios-de-la-selecciuxf3n-natural-1}

Tres principios se encuentran atrás de la propuesta de Darwin para dar
cuenta de la variabilidad y la adaptación de los rasgos de los
individuos, en particular de los cambios en la frecuencia relativa de
los rasgos en una población : 1. Variabilidad Existe variabilidad en
rasgos morfológicos, fisiológicos o conductuales entre miembros de una
especie. 2. La variabilidad es heredable 3. Hay una covarianza entre los
diferentes rasgos y el número de descendientes dejados por los
individuos, la covarianza es parcialmente atribuible al papel causal de
los rasgos. Si se satisfacen los tres principios que acabamos de
describir, de generación a generación, el rasgo con mayor éxito
reproductivo incrementará en frecuencia en la población.

\subsection{4. Determinantes del éxito
reproductivo}\label{determinantes-del-uxe9xito-reproductivo}

\subsubsection{1. Filtros}\label{filtros}

Dada cierta variabilidad en un rasgo, la distribución estadística de las
características del entorno es la que determina el éxito reproductivo
diferencial de dicho rasgo. Una analogía útil para entender el proceso
de selección natural es considerar a los entornos como filtros sobre la
variabilidad en los rasgos. Son estos filtros los que cambian la
frecuencia relativa de un rasgo en una población y determinan su éxito
reproductivo diferencial.

Un ejemplo muy sencillo de selección es el juego infantil de inserción
de cuerpos geométricos. La cubierta de la caja tiene un conjunto de
orificios de diferentes formas: cuadrados, círculos y triángulos. La
tarea para el infante es insertar en estos orificios a objetos con forma
de cilindros, cubos o pirámides. Para hacer uso de esta analogía de la
selección natural, imagine ahora una cubierta enorme de este tipo, con
una distribución de orificios de las tres formas, y por otro lado, un
saco con una población de los tres objetos con diferentes frecuencias
relativas. Esa distribución de objetos sería el equivalente a la
``generación número 1'' dentro del esquema de la selección natural.
Vaciamos el contenido del saco sobre la cubierta y observamos la
distribución de objetos que quedaron después de pasar por el filtro de
la misma. El número de objetos que sí se ajustaron a un lugar dentro de
la cubierta es la distribución de objetos en la segunda generación. Esa
distribución depende de dos factores: la distribución original -es
decir,el número original de cubos, cilindros y pirámides- y de la
distribución de los tres tipos de orificios en la cubierta. Si la
cubierta solo tuviese cuadrados y círculos, en la segunda generación
solo observaremos cilindros y cubos. Note que no se selecciona el mejor,
solo se eliminan los que no pasan a través de los filtros. Un problema
con el ejemplo anterior es que no contempla la creación de nuevos
objetos. La selección natural necesita de un proceso que, de generación
a generación, produzca nueva variabilidad.

\subsubsection{2. Determinantes de la
variabilidad}\label{determinantes-de-la-variabilidad}

La variabilidad puede cambiar intrínsecamente, de generación en
generación o puede cambiar por un factor externo que ocurre en una
oportunidad. Dos factores están detrás del origen de la variabilidad de
generación a generación. El primero son las \emph{mutaciones genéticas
aleatorias}, el segundo es la \emph{reproducción sexual}. El primero,
las mutaciones genéticas aleatorias, produce modificaciones que pueden
tener tres consecuencias: o bien ser letales para el individuo (haciendo
que este no pase por ningún filtro evolutivo adicional); o inducir un
mayor éxito reproductivo que los demás rasgos heredados (contribuyendo a
que el individuo atraviese un nuevo filtro) o bien pueden ser neutrales
y no tener ningún efecto sobre el éxito reproductivo. En este último
caso, el gen mutado se desliza a lo largo de generaciones y cambia la
frecuencia relativa de los diferentes genes: a este proceso se le conoce
como \emph{deriva genética} (vea el artículo de xxx para una descripción
más detallada). La reproducción sexual es la segunda gran fuerza de
variabilidad. La mitad de la conformación genética de cada descendiente
proviene del macho y la otra mitad de la hembra; adicionalmente, cada
una de las mitades proviene de un muestreo aleatorio de los genes del
macho y de la hembra.

La otra fuente de variabilidad es la que se genera cuando un accidente
geológico aísla a una población o la divide en subpoblaciones que no
pueden interactuar. Es el caso de una erupción volcánica, una separación
geológica que forma una isla o algo creado por el humano, como una barda
en medio del desierto. El \emph{aislamiento geográfico} produce que un
mismo acervo genético separado por una barrera geográfica pueda
resultar, mediante la recombinación aleatoria, en dos poblaciones con
rasgos diferentes. Dicho proceso inclusive puede derivar eventualmente
en la generación de dos especies diferentes que ya no pueden
reproducirse entre sí.

\subsubsection{En resumen,}\label{en-resumen}

\begin{enumerate}
\def\labelenumi{\arabic{enumi}.}
\item
  En la teoría de la selección natural los cambios en el éxito
  reproductivo diferencial se deben a las propiedades (filtros,
  restricciones) en el entorno de los organismos.
\item
  La variabilidad aleatoria de los rasgos junto con los cambios en el
  entorno son el motor de la evolución y de la adaptación de los rasgos
  de un organismo.
\item
  Como resultado de la operación de los tres principios de la selección
  natural, se observa un incremento gradual en el éxito reproductivo de
  ciertos rasgos en la población. Este incremento puede describirse como
  un proceso de ascenso de colina, que resulta en el ajuste de los
  rasgos a las propiedades estadísticas del entorno.
\item
  Reservamos el término de \emph{adaptación} a los rasgos resultado del
  proceso de selección natural. Es importante distinguir entre rasgos
  que son benéficos para un organismo en el presente de aquellos rasgos
  cuya existencia es el resultado de un proceso de selección natural.
\item
  Para entender la evolución de un rasgo, es necesario especificar en
  detalle los filtros, caracterizados como propiedades estadísticas del
  entorno. Entre los filtros más generales encontramos:

  \begin{enumerate}
  \def\labelenumii{\alph{enumii}.}
  \tightlist
  \item
    Limitaciones en recursos
  \item
    Competencia con organismos de la misma especie y de otras especies
  \item
    Selección sexual
  \item
    Selección dependiente de la frecuencia del rasgo
  \end{enumerate}
\item
  Una característica importante del proceso de selección natural es que
  con frecuencia actúa en entornos que son modificados por el mismo
  éxito reproductivo de una población. Adaptaciones a un nicho generan
  un nuevo nicho con un conjunto de nuevos filtros.
\item
  La teoría de la selección natural debe acompañarse de la
  especificación de las restricciones, genéticas y físicas, que limitan
  el tamaño del espacio de los posibles rasgos que son candidatos
  viables para un proceso de selección. Por ejemplo, físicamente hay una
  relación posible entre el peso de un animal y el diámetro de sus
  patas. Es decir, ciertos tamaños de patas en algunas especies no son
  rasgos viables para ser seleccionados por selección natural debido a
  las restricciones físicas propias de la anatomía del organismo. En ese
  sentido, la selección natural no produce la solución perfecta a un
  problema, sino que resulta en la mejor de las posibles soluciones dado
  un conjunto de restricciones.
\item
  Encontramos dos tipos de explicaciones evolutivas:

  \begin{enumerate}
  \def\labelenumii{\alph{enumii}.}
  \item
    Explicaciones Causales: tras la observación de un rasgo y de su
    posible función, estas explicaciones buscan encontrar los cambios en
    los entornos, las posibles restricciones y la historia de esos
    rasgos que pueden dar cuenta de su aparición en una población.
  \item
    Explicaciones de optimización: estas explicaciones están ancladas en
    las herramientas de la teoría matemática de la optimización y
    consisten en elaborar modelos del entorno (preferentemente
    matemáticos) como un problema y derivar su solución óptima dado un
    conjunto de restricciones. El éxito de estas explicaciones se
    sustenta en la calidad del modelo de las propiedades estadísticas
    del entorno que funcionan como filtros y que constituyen el problema
    a resolver, así como de la identificación completa de las posibles
    restricciones de las cuales se derivan las soluciones.
  \end{enumerate}
\item
  Sin embargo, no todos los rasgos observados son el resultado de un
  proceso de selección natural, hay otros factores que se combinan con
  el proceso de selección para poder entender la evolución de un rasgo.
  Estos factores están principalmente asociados con los procesos que
  resultan en la generación de variabilidad aleatoria.
\end{enumerate}

\bookmarksetup{startatroot}

\chapter{Evolución de la Adaptabilidad del Comportamiento: El papel de
las
restricciones}\label{evoluciuxf3n-de-la-adaptabilidad-del-comportamiento-el-papel-de-las-restricciones}

Si el término ``adaptación'' se utiliza para designar aquellos rasgos
que son resultado del proceso de selección natural, el concepto de
``adaptabilidad'' del comportamiento se define como la medida en que un
comportamiento contribuye al éxito reproductivo en el presente. El
primer término hace referencia al origen de un comportamiento, mientras
que el segundo hace referencia a la función que cumple un comportamiento
actualmente. En particular, la adaptabilidad de un comportamiento se
modifica naturalmente en función de cómo este aporta a dos
características clave de la vida de los organismos: el metabolismo y la
reproducción (Godfrey-Smith, 2017). En cuanto al metabolismo, los
sistemas biológicos gastan y agotan la energía almacenada y requieren de
un constante abastecimiento de ella. Los procesos evolutivos van
moldeando distintas formas exitosas de reabastecimiento a través del
éxito reproductivo diferencial del organismo.

El reabastecimiento de energía a través de la acción del organismo está
limitado por dos grupos de restricciones, uno del organismo y otro del
entorno. La primera condición limitante es que el comportamiento toma un
tiempo para llevarse a cabo. Dado que el tiempo disponible es finíto, la
suma de todos los comportamientos llevados a cabo es igual al total del
tiempo disponible.

\[
T = t_1 + t_2 + t_3 \dots t_n
\]

Esta es una \emph{restricción lineal} bajo la cual los diferentes
comportamientos \emph{compiten} por el tiempo disponible. Este
planteamiento asume que no existe tal cosa como una ``ausencia de
comportamiento'', por lo que en toda unidad de tiempo se está generando
un tipo de conducta que consume una duración particular de tiempo
(aunque este comportamiento sea ``reposar'' o que el organismo
permanezca estático). Bajo este esquema, el incremento de una unidad de
tiempo asignado a uno de los comportamientos por parte del organismo,
implica una menor unidad de tiempo disponible para el resto de los
comportamientos. Cada segundo de tiempo t dedicada al comportamiento 1
es una un segundo menos que puede dedicarse al resto de los \emph{n}
comportamientos.

Una segunda restricción del organismo es su estructura biológica
metabólica y neuronal, las cuales limitan el posible rango de
comportamientos y uso de recursos energéticos.

El éxito reproductivo de un organismo no solo depende de su habilidad
para reabastecerse de energía: es necesario, entre otras cosas,
encontrar una pareja con la cual reproducirse, evitar depredadores y
escapar del frío y/o calor extremo. Todas estas consecuencias tienen una
importancia biológica por su impacto sobre el éxito reproductivo de los
organismos y coinciden con lo que comúnmente se conoce como
``refuerzos'': nosotros seguiremos a Baum y, para enfatizar el origen y
el papel en la evolución de estos elementos, les llamaremos
\emph{sucesos biológicamente importantes SBI}. En la ecuación 1,
corresponde a las acciones ligadas a los diferentes SBI. Cuando hablamos
de SBI, no nos limitamos a lo que podría entenderse como necesidades
básicas. Selección natural ha operado para filtrar mecanismos que
permitan traducir éxito reproductivo diferencial en la detección de
todas las variables que pueden estar correlacionadas con él. Por
ejemplo, selección natural ha filtrado mecanismos de cognición social
que permiten a diversos organismos detectar si una acción particular
será bien recibida dentro de un grupo específico de su especie: si bien
la aceptación social no se traduce inmediatamente en un mayor acceso a
la reproducción o en mayores opciones para nutrir el metabolismo del
organismo, a la larga la aceptación social es un SBI dado que facilita
las condiciones que permiten al organismo un mayor acceso a ambos
elementos. Por lo tanto, para diversas especies, la cognición social es
un proxy del éxito reproductivo diferencial, aunque esta variable no
satisface directamente una necesidad básica en términos evolutivos.

El segundo grupo de restricciones son aquellas que describen la
disponibilidad de recursos de reabastecimiento y su relación a
propiedades del entorno de los organismos. A estas restricciones les
llamaremos las \emph{propiedades estadísticas del entorno.} Son estás
propiedades las que, en conjunción con la restricción lineal sobre el
comportamiento, determinan el posible conjunto de distribuciones de
acciones que han de ser sometidas al filtro de la selección natural.

Las condiciones del entorno biológicamente importantes pueden ser
relativamente constantes en el tiempo o variar a lo largo de él.
Dependiendo de esa variabilidad en el entorno, el proceso de selección
puede ocurrir en dos escalas temporales diferentes: puede darse entre
generaciones o durante la vida individual de un organismo. Si las
condiciones son constantes, el proceso de selección ocurre a lo largo de
generaciones y resulta en programas conductuales específicos que en la
Psicología se les ha conocido como reflejos, instintos o sesgos y de los
cuales se dice que son el resultado de un proceso evolutivo. Podemos
decir que el proceso evolutivo ``codifica'' genéticamente las
condiciones constantes del ambiente. Consideren la tarea de diseñar un
robot que se mueve en un entorno fijo y que tiene solo una tarea por
realizar. En ese caso, la solución del ingeniero consiste en
\emph{codificar} las características fijas del entorno en el software
del robot. En el curso, vamos a llamar \emph{comportamiento adaptado} al
que resulta de un proceso de selección natural: es decir, el
comportamiento resultante de un proceso de ajuste lento y gradual a las
condiciones del entorno a lo largo de generaciones. En ese sentido, el
comportamiento adaptado es un tipo de adaptación.

Sin embargo, los entornos de la mayoría de los organismos y de los
robots de servicio y exploración espacial son entornos variables,
volátiles e inciertos. Este tipo de entornos requieren de mecanismos que
se ajusten a la variabilidad en tiempo real (y no a lo largo de
generaciones) para poder ser afrontados con éxito. Por ejemplo, para
diseñar un robot (como Wall-E) que pueda funcionar en entornos con SBI
variables e inciertos, un ingeniero debe conocer y poder modelar la
disponibilidad de fuentes de energía para el reabastecimiento del robot,
la distribución de los SBI y la disponibilidad de opciones para que el
agente pueda llevar a cabo su meta (que sería recolectar y compactar
basura en el caso de Wall-E).

Los entornos variables, volátiles e inciertos se caracterizan por sus
propiedades estadísticas. Sin esta variabilidad, no habría selección de
mecanismos que permitan la adaptación de los organismos a entornos
cambiantes dentro del periodo de sus vidas individuales. La adaptación
del comportamiento de los organismos a las propiedades estadísticas del
entorno dentro de su lapso de vida individual, la atribuimos a un
proceso que llamamos ``Aprendizaje''.

En resumen,

\begin{enumerate}
\def\labelenumi{\arabic{enumi}.}
\item
  En entornos sin variabilidad, la selección natural puede resultar en
  comportamientos adaptados a las propiedades invariantes del entorno.
\item
  En entornos variables e inciertos, selección natural resulta en el
  surgimiento de mecanismos que permiten ``comportamiento adaptable'' y
  que llamaremos \emph{aprendizaje} (¡ojo! este concepto es distinto al
  de ``comportamiento adaptado'').
\item
  Selección natural opera en estructuras, genes y las expresiones de
  genes (como un sistema nervioso, por ejemplo). Estos son los elementos
  que permiten y subyacen a los mecanismos del comportamiento adaptable.
\end{enumerate}

En términos evolutivos, la tarea de los agentes es \emph{reducir la
incertidumbre} acerca de la ocurrencia de los SBI, \emph{predecir}
efectivamente su ocurrencia y llevar a cabo las \emph{acciones}
necesarias para interactuar óptimamente con estos. Así, un primer paso
para el estudio del comportamiento inteligente es tener una
especificación detallada de las propiedades estadísticas de los entornos
de los organismos. Cuatro propiedades estadísticas de los sucesos
biológicamente importantes han dominado el estudio del comportamiento
adaptable: el tiempo y el lugar de ocurrencia, la relación (correlación,
covarianza) con otras características del entorno y la relación con el
comportamiento de un organismo. El supuesto más importante que haremos
es que si las consecuencias relevantes en el entorno se distribuyen en
ciertos tiempos, lugares y están asociadas con ciertas estímulos y
comportamientos, un organismo que pueda detectar estas propiedades
estadísticas y ajustar su conducta a ellas, podrá asignar más
óptimamente su distribución de comportamiento a las metas en
competencia.

Los SBI ocurren de forma incierta y la tarea para los organismos es
descubrir si su ocurrencia está ligada a ciertos tiempos o a ciertos
lugares. Pueden también formar parte de una estructura causal, y en esos
casos, la tarea de los organismos es descubrir con qué propiedades del
entorno están vinculados. Adicionalmente, los SBI pueden depender de las
acciones de los agentes, en cuyo caso, la tarea es descubrir cuál acción
es responsable de su ocurrencia.

Una propiedad adicional de los entornos es que las cuatro posibles
regularidades que acabamos de detallar (características estadísticas del
entorno), pueden variar a su vez de acuerdo a otros estados externos del
mundo de los agentes (noche y día, escuela y casa). En los libros de
texto, a la adaptación de los organismos ante estas propiedades del
contexto se les estudia bajo el nombre de control de estímulos.

Un problema de adaptación más difícil es cuando las propiedades
estadísticas del entorno cambian sin señales externas, como resultado de
estados ocultos al agente. En estos casos, la tarea es determinar si el
cambio observado en la propiedad estadística del entorno constituye un
cambio aleatorio o si realmente se trata de una modificación en el
estado particular del mundo. En la literatura clásica, a este problema
se le ha estudiado bajo el nombre del fenómeno de la extinción.

Hasta el momento hemos supuesto, al igual que la gran mayoría de los
teóricos hasta los años 60s del siglo pasado, que los organismos solo
detectan sucesos individuales. Revisaremos evidencia de que en adición,
los organismos detectan y se adaptan a una característica de segundo
orden de los entornos: la tasa de ocurrencia de sucesos individuales en
su entorno, la cual se define como el número de sucesos por unidad de
tiempo. Por ejemplo, las abejas seleccionan su tiempo de estancia en
diferentes lugares de un jardín en función de la tasa de encuentro con
flores con néctar. Veremos que la noción de tasa de ocurrencia juega un
papel muy importante en las explicaciones contemporáneas del
comportamiento.

Finalmente, podemos considerar una propiedad de los entornos que puede
considerarse de tercer nivel. Nos referimos a la incertidumbre acerca
del tiempo, lugar, covarianzas y tasas de los SBI. Los camiones de
transporte público no pasan siempre a la misma hora (incertidumbre de
tiempo), en el mercado no siempre se encuentra la misma fruta
(incertidumbre de lugar), no llueve siempre que está nublado
(incertidumbre de covarianzas) y un jugador de fútbol no alcanza siempre
la misma tasa de goleo en una temporada (incertidumbre de tasas). Hay
una relación que podemos describir como una probabilidad condicional en
el sentido de que la probabilidad de un evento en sí misma depende (o se
encuentra condicionada) a otra distribución de probabilidad. Esta
descripción nos sirve para distinguir entre cuando la incertidumbre de
un evento es esperada o inesperada. La incertidumbre es esperada si
existe una única distribución de probabilidad que la describa. Es el
caso cuando se lanza una moneda sin imperfecciones al aire, donde
sabemos que la probabilidad de que caiga águila o sol es de 0.5. Por
otro lado, la incertidumbre es inesperada cuando los parámetros de la
distribución cambian de acuerdo a una segunda distribución de
probabilidad: en este caso se habla de una probabilidad condicional. Un
caso sería una moneda cuya probabilidad de que caiga águila cambia a lo
largo del tiempo de acuerdo a otra probabilidad, como la distribución de
la velocidad del viento en un entorno, por ejemplo.

En conclusión, si asumimos que la teoría de la selección natural es una
descripción correcta y si los organismos y el entorno operan bajo las
restricciones descritas, podemos concluir que el objeto de estudio
natural de la Psicología es el estudio de la adaptabilidad del
comportamiento. En otras palabras, la Psicología estudia distintos
comportamientos y los explica en función de su capacidad para brindarle
acceso al organismo a sucesos que le son biológicamente significativos.

En el curso veremos cuales son las soluciones óptimas a los diferentes
problemas de adaptación generales que encaran los organismos, así como
los distintos mecanismos que posibilitan alcanzar tales soluciones a
distintas especies. Veremos que para lograr el objetivo de entender el
comportamiento adaptable es fructífero iniciar con un detallado análisis
del problema de adaptación en cuestión. Esto último requiere modelar
tanto las propiedades estadísticas del problema como las restricciones
que imponen los posibles mecanismos de las distintas especies sobre la
solución óptima.

\bookmarksetup{startatroot}

\chapter{Asignación de Crédito}\label{asignaciuxf3n-de-cruxe9dito}

Al interactuar con su entorno, un agente se encuentra con un constante
flujo de estímulos y respuestas que se despliegan en el tiempo. Algunos
de los sucesos que encuentra son biológicamente significativos,
importantes para su éxito reproductivo. El encuentro inesperado con un
suceso biológicamente importante echa a andar dos mecanismos: Uno que
controla la respuesta inmediata al SBI y un segundo mecanismo que
permite predecir su futura ocurrencia. Considere un organismo que
encuentra un inesperado pedazo de comida o un depredador. El primer
mecanismo le permite al organismo manipular y consumir la comida, o huir
y escapar del depredador. El segundo mecanismo, el que posibilita
predecir y controlar un SBI, implica la existencia de una estructura
causal en el entorno del organismo: esto es, que existen sucesos que
predicen o respuestas que producen la comida o evitan al depredador. La
tarea para el agente es seleccionar, dentro de un número gigantesco de
posibilidades, a cuál suceso o respuesta atribuirle la ocurrencia de un
SBI. A este problema de adaptación se le conoce como el de la
\emph{asignación de crédito}.

El vasto espacio de posibles candidatos para la asignación de crédito de
un SBI incluye la hora a la que ocurre, dónde ocurre, el enorme grupo de
sucesos que lo acompañan o los comportamientos que un organismo genera;
pero también podemos incluir momentos, espacios, sucesos y
comportamientos que ocurrieron en cualquier momento previo. La comida
que un perro callejero se encuentra en una banqueta puede deberse a un
transeúnte que el perro percibe en ese momento alejándose de la comida,
o a miles de posibles transeúntes que la tiraron en un tiempo cada vez
más alejado de su encuentro con el alimento, pero pudo deberse a alguien
que la tiró desde un transporte público un segundo antes, o diez minutos
antes o un día antes.

Para hacer más manejable la asignación de crédito ante las limitaciones
de nuestras observaciones y la riqueza de candidatos, selección natural
filtró mecanismos que llamamos \emph{''sesgos inductivos'',} los cuales
logran dos cometidos: primero, reducen el espacio de candidatos a
asignación de crédito y, segundo, establecen un orden de evaluación para
poner a prueba a los candidatos del espacio más reducido en un momento
posterior. Los sesgos pueden ser el resultado de la codificación
genética de propiedades del entorno bajo el cual evolucionó la especie
del organismo o el resultado de su experiencia y aprendizaje
individuales.

Históricamente, la contigüidad entre sucesos fue el primer sesgo en
recibir atención. El sesgo consiste en suponer que la ``contigüidad''
entre un estímulo o una respuesta y un SBI es una regla evolutiva muy
útil para reducir el espacio de opciones de asignación de crédito. El
espacio de asignación de crédito se reduce a sólo aquellos eventos
contiguos con el SBI. Si al momento que el perro callejero encontró la
comida, este prestaba atención a una ambulancia que pasaba con la sirena
encendida y a un transeúnte vestido como estudiante universitario, su
espacio de asignación de crédito se reduciría a esos dos sucesos. Para
seleccionar entre ellos dos, operaría un segundo sesgo que veremos en
una sección subsecuente.

Al inicio del siglo XX, Pavlov le dio sentido experimental y conceptual
al estudio de este sesgo. El propósito de los experimentos de Pavlov fue
establecer la importancia de la contigüidad en la formación de nuevas
asociaciones entre estímulos previamente neutrales y respuestas
reflejas. El protocolo, representado en la Figura x, consistió en
presentarle a un perro un estímulo auditivo seguido por acceso a comida.
Pavlov midió la salivación ante la comida y ante el estímulo auditivo,
antes y después de haber sido presentado junto con la comida. Encontró
que aparear el sonido a la comida, resultó en que el perro salivaba
ahora no tan solo a la comida, sino también al sonido. Al sonido se le
conoce como \emph{estímulo condicionado EC } y a la comida como
\emph{estímulo incondicionado EI}.

En los primeros protocolos experimentales se consideraba solo un
candidato al cual asignar crédito (como un tono) y la manipulación
experimental era una imprecisa medida de contigüidad que implicaba
diferentes relaciones temporales entre el EC y el EI. Los siguientes son
los protocolos más empleados: Ver Figura.

En estos protocolos se encontró que la medida de condicionamiento
disminuye conforme incrementa el tiempo entre la terminación del EC y el
inicio del EI. A esta relación se le llamó el \emph{gradiente de la
demora}. Dependiendo de la preparación, después de menos de un minuto de
intervalo entre EC y EI no se observaba aprendizaje. Adicionalmente, si
el EI se presentaba antes del EC (procedimiento huella) no se observaba
aprendizaje. Más adelante veremos que la historia es más compleja que
este resumen, pero por el momento es suficiente que se tenga claridad
sobre estos resultados.

\subsection{¿Es la contigüidad una condición necesaria para el
aprendizaje?}\label{es-la-contiguxfcidad-una-condiciuxf3n-necesaria-para-el-aprendizaje}

Nos podemos preguntar si la contigüidad es el único sesgo que reduce el
espacio de candidatos a la asignación de crédito. Para darle respuesta a
esta pregunta se utilizan dos estrategias: la primera consiste de
protocolos experimentales en los que dos o más estímulos igualmente
contiguos con el SBI compiten por la asignación de crédito. Este
protocolo nos permitiría demostrar si la contigüidad es un factor
\textbf{suficiente} para reducir el espacio de candidatos en asignación
de crédito: en el sentido de que, si ambos estímulos son igualmente
contiguos, pero el organismo sólo aprende sobre uno de ellos, esto
demostraría que la contigüidad no es una condición suficiente para el
aprendizaje. La segunda estrategia se trata de protocolos en los cuales
se modifica la demora de la presentación del SBI para observar si la
asignación de crédito se mantiene. Esta clase de protocolo nos
permitiría darle respuesta a la pregunta de si la contigüidad es una
condición \textbf{necesaria} para el aprendizaje. John García condujo
justo estos experimentos. Inicialmente, a partir de una observación
accidental trabajando con los efectos de radiación sobre ratas, García
encontró que las ratas dejaban de comer y generaban una aversión a su
dieta habitual a pesar de que el efecto de la radiación se presentaba
mucho tiempo después de la ingesta de la comida.

La figura x muestra el protocolo del experimento de García. A todos los
animales se les daba acceso a un bebedero con agua azucarada en el cual
cada contacto detonaba la presentación de un tono. De esa forma había un
compuesto conformado por dos estímulos: un tono (EC) y el agua dulce
(EI). A la mitad de los animales se les daba una descarga eléctrica con
cada lengüetazo que daban al bebedero, mientras que a la otra mitad de
los animales se les inyectaba una sustancia que producía un malestar
estomacal. García encontró que las ratas que recibieron las descargas
eléctricas no dejaron de beber el agua azucarada, pero sí evitaban tocar
el bebedero cuando este producía el tono; mientras tanto, las ratas con
malestar estomacal dejaban de beber el agua dulce, pero no presentaban
aversión al tono. Este experimento muestra que la naturaleza del SBI
determina los elementos que entran en el espacio de asignación de
crédito. Para las ratas, igual que para otras especies omnívoras, como
la nuestra, cuando el SBI es un malestar estomacal, el espacio de
elección está conformado por elementos con sabor, pero no por elementos
visuales o auditivos. Al sentirnos mal del estómago, lo primero que
hacemos es buscar qué comimos, aunque nuestra última comida haya sido
muchas horas antes. A este sesgo se le conoce como el sesgo de
\emph{relevancia biológica}.

Las ratas aprenden a evitar el sabor asociado con enfermedad aun cuando
existen largas demoras (horas) entre la experiencia del sabor y la
presentación de la enfermedad. Sin embargo, el que la \emph{contigüidad
no sea necesaria}, no significa que no sea un factor. En subsecuentes
experimentos que manipularon la duración entre el consumo del alimento y
la enfermedad, se encontró también un gradiente de demora en el cual la
aversión aprendida al sabor incrementa en función de la reducción de los
intervalos entre la presentación del alimento y el EI. Una evidencia
adicional sobre el papel de la contigüidad la encontramos en estudios
que presentan al organismo dos sabores antes de que este atraviese su
experiencia de enfermedad. En estos estudios se ha encontrado que la
aversión se genera al sabor que es temporalmente más cercano a la
sensación de malestar.

Usando la misma preparación de aversión a sabores de García, se encontró
otro sesgo importante que determina cuál de los elementos en el espacio
de candidatos a la asignación de crédito es considerado primero. En
experimentos en los que se presentan dos sabores, uno novedoso y otro
familiar, ambos igualmente contiguos con la enfermedad, las ratas
aprenden a evitar solo el sabor que era novedoso. A este sesgo se le
conoce como el sesgo de la \emph{novedad}.

El sesgo de la relevancia biológica es evolutivo. Para especies como la
rata que son omnívoras y viven principalmente en la oscuridad es
importante detectar qué alimento es tóxico por su sabor. Otras especies
como las palomas, que habitan nichos ecológicos diferentes, no generan
aversión a los sabores. Para estas especies, la dimensión relevante es
la estimulación visual y no el sabor del alimento. La coevolución entre
aves y polillas ejemplifica la importancia de la relevancia biológica.
Las polillas son un alimento para ciertas aves; por otro lado, la
selección natural resultó en algunas especies de polillas que son
tóxicas para las aves. Esta toxicidad es identificable a través de
señales visualmente perceptibles, gracias a lo cual, las aves pueden
desplegar su sesgo de relevancia biológica hacia los estímulos visuales
y aprender a evitar este tipo de polilla. Simultáneamente, otro grupo de
polillas no tóxicas evolucionaron para tomar ventaja de ese mismo sesgo
de las palomas y desarrollaron patrones visuales similares a los de las
especies tóxicas para evitar ser depredadas. Poner figura.

En resumen, los estudios de aversión a sabores sugieren que: 1.
Contigüidad no es una condición necesaria para el aprendizaje. 2. Sin
embargo existe un gradiente temporal y hay una mayor aversión al sabor
más cercano al malestar estomacal. 3. Existen sesgos biológicos que
generan una predisposición a considerar sólo ciertos estímulos para
asignación de crédito, los cuales dependen del suceso biológicamente
importante, como por ejemplo, sabor para enfermedad estomacal en
omnívoros y estímulos visuales para aves. 4. Un importante sesgo
adicional es priorizar sucesos que son novedosos (o sorprendentes)
dentro del proceso de asignación de crédito. 5. La contigüidad es uno de
los sesgos, pero no constituye una condición necesaria para el
aprendizaje.

\subsection{¿Es la contigüidad una condición suficiente para el
aprendizaje?}\label{es-la-contiguxfcidad-una-condiciuxf3n-suficiente-para-el-aprendizaje}

A finales de los años 60s del siglo pasado, un grupo de investigadores,
entre los que destacan Leon Kamin, Robert Rescorla y Allan Wagner,
condujeron un grupo de experimentos dirigidos a darle respuesta a la
pregunta sobre si la contigüidad es una condición suficiente para el
aprendizaje. En estos experimentos se presentó un compuesto de dos o más
estímulos (condicionados), igualmente contiguos con el suceso
biológicamente importante, el llamado estímulo incondicionado (EI). Un
ejemplo de un compuesto de estímulos es la presentación simultánea de
una luz y un tono, o la combinación de una figura visual y un color.

\subsubsection{Ensombrecimiento}\label{ensombrecimiento}

Los sucesos que anteceden a un suceso biológicamente importante
regularmente están compuestos de estímulos que varían en diferentes
dimensiones. Un perro que los amenaza, no solo ladra y gruñe, tiene
también cierto color, ciertos ojos y cierta boca. Si les llegara a
morder, todas estas características del perro estarían contiguas con el
suceso aversivo de la mordida. Si la contigüidad fuese suficiente para
el aprendizaje, todas y cada una de las características del perro se
convertirían en predictores de un ataque. Reynolds puso a prueba esta
conjetura con un sencillo experimento. A dos palomas se les entrenó a
discriminar entre dos teclas a las que podían picar. Una de las teclas
generaba acceso a un comedero, la otra no. Las teclas estaban iluminadas
por un compuesto de dos estímulos que variaban en color o forma. La
tecla positiva era un triángulo blanco sobre un fondo rojo, la tecla
negativa era un círculo blanco en un fondo verde. (Ver figura). Después
de que los animales habían aprendido a responder solo a la tecla
positiva, se le presentaron los cuatro estímulos por separado. Se
encontró que las palomas responden solo a uno de los dos estímulos del
compuesto positivo. Una paloma respondía a la figura, la otra al color.

La importancia del experimento radica no solo en la demostración de que
la contigüidad no es una condición suficiente, sino en la ilustración de
un principio que será clave en el curso: la \emph{competencia} entre
elementos, sean estímulos o respuestas. El experimento de Reynolds
ilustra que los estímulos presentados en forma simultánea dentro de un
compuesto compiten entre ellos por la asignación de crédito del
organismo. En ese sentido, la asignación del crédito a uno de los
estímulos por parte del organismo implica la no asignación de crédito al
otro estímulo presente. Retomando nuestro ejemplo, si el perro los
ataca, para algunos de ustedes el predictor del ataque será el gruñido,
para otros será el color y para otros será la raza. Cuando es la primera
experiencia con el compuesto de estimulación, los factores que
determinan cuál elemento gana incluyen la sobresaliencia de los
estímulos y su novedad. La siguiente pregunta es si la historia del
organismo con uno de los elementos del compuesto afecta la asignación de
crédito. A continuación, veremos una serie de experimentos que sugieren
que una vez que se asignó el crédito a un elemento, los organismos dejan
de considerar a otros elementos como candidatos.

\subsubsection{Bloqueo}\label{bloqueo}

Imaginen que, después de un par de experiencias visitando restaurantes,
ustedes aprenden que un mantel de tela es un buen predictor de la
calidad de la comida de un lugar. En su visita a un nuevo restaurante,
las mesas de este tienen manteles de tela, pero adicionalmente el
restaurante tiene música clásica de fondo. La calidad de la comida es
igualmente buena a la del último restaurante con manteles de tela que
visitaron, pero en este caso la comida fue contigua tanto con el mantel
de tela como con música clásica. ¿Habrán aprendido que la música clásica
es un predictor de la buena comida? Para darle respuesta a esta
pregunta, tendrían que observar si al verse forzados a escoger entre dos
restaurantes sin manteles de tela, seleccionarán aquel que tiene música
clásica sobre el que no la tiene. Veremos que los experimentos indican
que una vez que se asignó el crédito de un SBI a un elemento de un
compuesto, los otros elementos del compuesto no adquieren ningún
crédito.

En 1969 Kamin corrió el primer experimento evaluando la intuición
anterior. A dos grupos de ratas se les presentó un compuesto de luz y
tono seguido de una descarga eléctrica. Ver Figura. Para el grupo
experimental, en una fase anterior se le presentaba la luz seguida de la
descarga eléctrica. En la tercera fase, de prueba, se le presentaba el
tono sin la luz para evaluar qué tanto habían aprendido las ratas acerca
de él. Noten que para los dos grupos, el tono antecede a la descarga
eléctrica. La única diferencia entre los dos grupos fue la experiencia
previa de la luz con la descarga eléctrica. Kamin encontró que a pesar
de que para los dos grupos el tono aparecía contiguo con la descarga
eléctrica, las ratas del grupo con el entrenamiento luz - descarga
eléctrica no mostraron evidencia de que el tono recibiera ningún crédito
por la presentación de la descarga eléctrica. Se dice que la experiencia
con la luz bloquea el aprendizaje acerca del tono. De la misma forma, en
nuestro ejemplo previo, el mantel de tela bloqueaba el aprendizaje
acerca de la música clásica. Estos experimentos muestran que el grado de
aprendizaje acerca del elemento de un compuesto seguido de un SBI,
depende del grado de aprendizaje adquirido previamente por el otro
elemento del compuesto. Una forma de interpretar estos resultados es que
los elementos compiten por la asignación de crédito en función de si uno
de ellos ya es un predictor del suceso biológicamente importante. El
fenómeno de bloqueo es evidencia adicional de que la contigüidad entre
un estímulo y un refuerzo no es una condición suficiente para el
aprendizaje.

\bookmarksetup{startatroot}

\chapter{Asignación de Crédito para
Respuestas}\label{asignaciuxf3n-de-cruxe9dito-para-respuestas}

El acceso a sucesos biológicamente importantes es fundamental para la
supervivencia y reproducción de los organismos. Aquellos organismos que
puedan predecir confiablemente la ocurrencia de los SBI tienen una
ventaja comparativa en términos de su éxito reproductivo. Aprender que
un cielo encapotado predice una fuerte lluvia le permite a un individuo
anticiparse y prepararse correctamente para ella. De igual forma,
escuchar un rugido le permite a una presa prepararse para el caso de un
posible ataque. Sin embargo, el individuo no tiene control sobre lo
nublado del cielo, ni sobre la presencia del depredador dado el rugido.
Puede predecir cuándo lloverá, pero no puede alterar el que llueva;
puede predecir que detrás del rugido esté un depredador, pero no puede
modificar su presencia.

Uno de los saltos importantes en la historia evolutiva fue la emergencia
de mecanismos biológicos que, a través de la acción y la interacción con
el entorno, permiten a los organismos \emph{controlar} la ocurrencia de
sucesos biológicamente importantes. Estos mecanismos se encuentran
estrechamente asociados a un componente específico de la estructura
causal de los entornos: las relaciones que describen cuáles acciones de
un organismo son exitosas para obtener mayores opciones de acceso a SBI.
Un ejemplo en nuestra especie de estas \textbf{''elaciones que describen
las acciones exitosas para acceder a mayores opciones de SBI''} son los
contratos laborales: en estos se contienen las reglas que especifican
las acciones a seguir para acceder a un monto de dinero (lo que equivale
a mayores opciones de SBI para nuestra especie). Otros ejemplos son: las
reglas que especifican qué acciones llevar a cabo si se desea tomar un
transporte público; las reglas que definen las acciones requeridas para
iniciar una relación amorosa; las reglas que especifican a su mascota
qué acciones le otorgan una comida especial; las reglas que especifican
a cada especie las acciones que facilitan su acceso a alimentos, así
como los actos que les permiten escapar y evitar a sus depredadores.

Desde la psicología, nos preguntamos cómo un organismo logra reconocer
dichas estructuras causales: específicamente, cómo puede determinar qué
acción específica, entre muchas posibilidades, es la responsable del
resultado deseado (el SBI). En los libros de texto, al estudio de la
respuesta a esta pregunta se le conoce como \emph{condicionamiento
instrumental o condicionamiento operante}. En estas notas abordaremos su
estudio con base en el mismo grupo de principios con los que abordamos
los resultados de los protocolos de condicionamiento clásico.

Antes de describir cómo se aplican los mismos principios a los fenómenos
de condicionamiento instrumental, es conveniente revisar el estudio
original del que surgió esta área de investigación: siguiendo el mismo
proceder que seguimos para entender el condicionamiento clásico. Al
inicio del siglo XX, Edward Thorndike condujo una serie de estudios con
gatos. Él diseñó una variedad de cajas experimentales, de las que un
gato encerrado podría escapar activando dispositivos como un cerrojo o
una palanca. (Ver figura). La medida del aprendizaje era el tiempo que
le tomaba al gato para escapar de la caja. De los datos mostrados en la
figura x, puede verse que el tiempo que le tomaba escapar al gato
disminuyó conforme aumentaba el número de ensayos en los que se le
encerraba. Al inicio, los gatos intentaban un número grande de
respuestas hasta que accidentalmente operaban el dispositivo que abría
la puerta. Después de algunos ensayos, el gato empezaba a activar el
dispositivo de escape inmediatamente después de que se le metía a la
caja. Thorndike caracterizó esta ejecución como una de \emph{ensayo y
error}. El gato intentaba diferentes respuestas (ensayos) y las
descartaba si no lo llevaban a salir de la caja (error).

Es posible identificar que el resultado de los experimentos de Thorndike
está compuesto de dos observaciones. La primera es el conjunto de
respuestas que lleva a cabo el gato antes de emitir la respuesta
correcta. La segunda observación es que después de varios ensayos, el
gato ejecuta de forma casi exclusiva e inmediata la respuesta que fue
exitosa para escapar de la caja. Para entender estas dos observaciones,
recordemos que en el capítulo anterior vimos que el encuentro inesperado
con un suceso biológicamente importante (SBI) echa a andar dos
mecanismos: uno que controla el comportamiento apropiado para la
interacción con y búsqueda adicional del SBI, y un segundo, que permite
predecir y controlar su futura ocurrencia.

Para analizar los dos principios que ilustra el comportamiento de ensayo
y error, recordemos que los sesgos inductivos pueden dividirse en dos
clases:

\begin{enumerate}
\def\labelenumi{\arabic{enumi}.}
\item
  Aquellos que determinan qué elementos -en nuestro caso respuestas-
  conforman el espacio de candidatos a la asignación de crédito.
\item
  Los sesgos que determinan cuál elemento dentro del espacio se debe
  considerar primero.
\end{enumerate}

En el caso del primer sesgo, el que delimita el espacio de respuestas
candidato a la asignación de crédito, las respuestas inducidas por el
SBI juegan un papel equivalente al de las mutaciones y la recombinación
genética dentro del proceso de generación de variabilidad en la teoría
de la evolución. Las respuestas del organismo y la variabilidad genética
coinciden en que ambas generan el espacio de opciones seleccionables
(candidatos) dentro de los procesos de selección de los que forman
parte. En la teoría de evolución, un conjunto de genes creado por las
mutaciones y la recombinación genética es sometido a un proceso de
selección por los cambios en el entorno; en la la teoría de los sesgos
inductivos, un conjunto de respuestas generadas por un organismo en su
interacción con el entorno es sometido a un proceso de selección por el
sesgo inductivo del organismo. Por otra parte, el segundo sesgo
referido, aquel que establece el orden de prioridad para evaluar las
respuestas candidato, es equivalente a los procesos específicos de
selección natural. De la misma forma en la que la selección natural se
dan procesos bien definidos para descartar y conservar genes
particulares de entre un amplio espacio de candidatos, existen procesos
bien definidos a nivel de los sesgos inductivos (de la segunda clase)
del organismo, que describen cómo este prioriza, descarta y conserva las
respuestas de entre su espacio de candidatos.

Para entender los resultados de sus experimentos, Thorndike propuso un
principio que se conoce como \emph{La ley del Efecto}, la cual establece
que: \emph{``En la presencia de un estímulo (situación, contexto) pueden
ocurrir una multitud de respuestas. Aquella que vaya seguida de un
estado de cosas satisfactorio tendrá que ser la que se asocia (conecta,
selecciona) con el estímulo.''}

La ley del efecto de Thorndike prioriza a la contigüidad como el factor
que determina tanto el espacio de candidatos, como el orden que
establece cuáles elementos evaluar primero. La ley no toma en cuenta el
origen de las respuestas que anteceden al ``estado de cosas
satisfactorio''. Este último término al poco tiempo se convertiría en el
concepto que hoy conocemos como ``refuerzos'' y que en estas notas
llamaremos también SBI. En lo que resta del capítulo, revisaremos la
historia y la evidencia acerca del papel de la contigüidad en la
asignación de crédito para una respuesta.

En 1947, Skinner publicó los resultados de un pequeño experimento para
demostrar la suficiencia de la contigüidad para el aprendizaje de
respuestas. A las palomas hambrientas se les presentó comida cada 15
seg., independientemente de su comportamiento. Se observó que a pesar de
esto, muchas palomas desarrollaron comportamientos estereotipados, como
girar en círculos o picotear ciertas áreas. Los comportamientos difieren
de paloma a paloma. (Ver figura). Skinner explicó estos resultados,
señalando que para cada ave, una respuesta ocurría de forma accidental
inmediatamente antes del refuerzo y esa contigüidad era responsable del
fortalecimiento de dicha respuesta. A partir de este tipo de
observaciones, Skinner concluyó: ``Decir que un reforzador es
contingente sobre una respuesta no significa otra cosa que decir que
``se presenta después de la respuesta''. Para Skinner, presumiblemente,
el condicionamiento ocurre únicamente debido a la relación temporal,
expresada en términos de la proximidad entre la respuesta y el
reforzador.

La sencilla historia anterior fue rápidamente cuestionada en una réplica
del estudio de Skinner, publicada por Staddon y Simmelhag en 1971. Al
igual que Skinner, a un grupo de palomas se le dió acceso a comida cada
15 segundos, independientemente de su comportamiento. A diferencia de
Skinner, estos investigadores observaron cuidadosamente el
comportamiento de las palomas a lo largo de los 15 segundos. Los
resultados se muestran en la figura x. No encontraron evidencia de que
se aprendiera la respuesta individual que accidentalmente antecede el
acceso a la comida. No obstante, observaron que para todas las palomas,
el comportamiento desplegado se podía agrupar en dos clases: una de
respuestas que ocurrían al final del intervalo, a las que llamaron
``respuestas terminales'', y las cuales incluían, entre otras, el
orientarse hacia la pared del comedero; y una segunda clase de
respuestas que agrupaba comportamientos que ocurrían a la mitad del
intervalo, a las que llamaron ``respuestas interinas'', entre las cuales
se observó la conducta de picar el piso. El estudio se replicó con ratas
y en ese caso también se observó la agrupación de respuestas en dos
clases.

De los resultados del experimento de Staddon y Simmelhag pueden
extraerse dos conclusiones, una negativa y otra positiva. Primero, en
relación al tema de este capítulo, podemos concluir que en los
experimentos en los que no existe una relación netamente causal entre
respuesta y SBI, el estímulo no selecciona a la respuesta que
accidentalmente le antecede: contradiciendo a la idea que la contigüidad
es una condición suficiente para el aprendizaje de respuestas. La
conclusión positiva es que la mera presentación de un SBI induce un
conjunto tipificado de respuestas, y que la periodicidad de la
presentación del SBI organiza el comportamiento de los organismos
alrededor del tiempo. En otro capítulo revisaremos en detalle otros
resultados relacionados y su papel en una teoría general del
comportamiento.

\subsection{¿Es la contigüidad entre una respuesta y un refuerzo una
condición necesaria para la adquisición de la
respuesta?}\label{es-la-contiguxfcidad-entre-una-respuesta-y-un-refuerzo-una-condiciuxf3n-necesaria-para-la-adquisiciuxf3n-de-la-respuesta}

En el experimento de superstición de Skinner no había una relación de
dependencia causal entre respuesta y refuerzo y Skinner buscaba
demostrar que la mera contigüidad era suficiente para el aprendizaje de
respuestas. Pero esta demostración dependía de la manera en la que se
especifica el concepto de contigüidad en términos concretos (¿cuándo
podemos considerar que un suceso es realmente contiguo? Si el SBI ocurre
un segundo antes del EC o dos segundos antes, ¿sigue siendo
contiguo?¿cuándo comienza y cuándo termina la contigüidad?). Para poder
estudiar sistemáticamente el papel de la contigüidad, es necesario
especificar la ventana temporal que define a dos eventos como contiguos.
La estrategia teórica-experimental inicial en esta área fue considerar
como contigüidad una ventana de cero segundos y considerar el impacto de
ventanas mayores como distintas instancias de efectos de la demora en el
refuerzo.

Para poder analizar el efecto de diferentes demoras, se requiere poder
controlar esa relación. Con esa finalidad, es necesario estudiar
protocolos en los que exista una relación de dependencia entre la
respuesta y el SBI, en particular, protocolos en los cuales se varíe el
tiempo entre las respuestas y la presentación de los SBI que son
generados por estas respuestas. A continuación revisaremos el efecto de
variar el valor temporal de los intervalos entre respuestas y los SBI
producidos por ellas.

En un primer experimento, Dickinson y sus colaboradores evaluaron el
impacto de diferentes demoras entre la respuesta de apretar una palanca
y la presentación del SBI. El experimento se condujo con ratas sin
ninguna experiencia previa con el procedimiento. El propósito del
experimento fue evaluar si las ratas aprenderían a apretar la palanca.
Cada respuesta producía un SBI con una demora fija. Durante el periodo
de demora, las ratas podían volver a responder y producir otro SBI con
una demora igual. Noten que con este procedimiento podía darse el caso
de que accidentalmente una de las respuestas de las ratas durante uno de
los periodos de demora ocurriera justo antes del SBI. Para descartar el
aprendizaje de respuestas por la mera contigüidad accidental con el SBI
(esto es, en ausencia de un efecto causal sobre este) había que evaluar
si el aprendizaje de la respuesta en las ratas se genera
independientemente de la existencia de una dependencia causal entre
respuesta y refuerzo. Para descartar esta posibilidad, se estableció un
grupo diferente de ratas a las cuales se les entregaba el SBI al mismo
tiempo que lo recibía el grupo dependiente: para este segundo grupo, sus
respuestas no tenían ningún efecto sobre el momento de aparición de los
reforzadores. Si ambos grupos mostraban los mismos patrones de
respuesta, eso significaba que la contigüidad temporal era el factor
determinante del aprendizaje de las respuestas y que las ratas del
primer grupo no aprendían en función del poder causal que identificaban
en sus respuestas. Esto también implicaría que las ratas del primer
grupo, el grupo dependiente, estarían aprendiendo a asociar la
presentación de los SBI con sus respuestas que se generaban accidental y
tardíamente dentro del intervalo de demora. La figura x muestra los
resultados del experimento. Se probaron tres valores de demora adicional
a la condición de contigüidad estricta. La medida empleada para
determinar si se había aprendido la respuesta fue el número de
respuestas de apretar la palanca por minuto. En el panel izquierdo puede
verse que la tasa de respuestas va decreciendo conforme incrementa la
demora hasta alcanzar un valor de 32 segundos entre respuesta y
reforzador, y cero aprendizaje con demora de 64 segundos. En el panel de
la derecha, se muestra el efecto de la dependencia respuesta-SBI sobre
el aprendizaje de las respuestas de las ratas a lo largo de 20 sesiones.
Se compara la tasa de respuesta para el grupo con dependencia
respuesta-SBI con la del grupo que recibía el SBI independientemente de
las respuestas de las ratas. Los resultados indican que para el segundo
grupo de ratas, la tasa de respuesta fue casi cero para todas las
demoras evaluadas, es decir no hubo aprendizaje, mientras que para el
primer grupo de ratas se observaron los patrones de aprendizaje con
demoras descritos previamente.

Como ya se mencionó, un problema a resolver para el experimento anterior
era controlar la posibilidad de que los organismos aprendieran
respuestas que aparecieran accidentalmente contiguas al SBI. Lattal y
Gleeson realizaron un experimento con una ingeniosa estrategia alterna
para controlar esta posibilidad. En su experimento, la respuesta de las
palomas de picar una tecla detonaba una demora de 10 seg. después de la
cual obtenían un refuerzo de comida; sin embargo, esto último sólo
ocurría si las palomas no daban ninguna respuesta durante el periodo de
demora. De esta forma, el diseño experimental garantizaba una demora
real de 10 segundos. En la Figura x, puede verse que aún eliminando la
posibilidad de una contigüidad accidental, las palomas aprenden la
respuesta de picar la tecla.

\subsection{Percepción de la relación de causalidad respuesta -
refuerzo}\label{percepciuxf3n-de-la-relaciuxf3n-de-causalidad-respuesta---refuerzo}

Los resultados que hemos reportado en este capítulo, nos llevan a
considerar la siguiente pregunta: ¿Pueden los organismos discriminar
entre refuerzos producidos por su comportamiento de aquellos que son
independientes de él?

Con un ingenioso experimento, Killeen pretendió explorar esta pregunta
de manera directa. En el experimento, la tarea para las palomas era
discriminar si un cambio en la iluminación de una tecla era el resultado
o no de su comportamiento. La respuesta de picar una tecla central
iluminada tenía como consecuencia el que, de manera aleatoria, cinco de
cada 100 respuestas (probabilidad de 0.05) causaran que se apagara la
tecla central y se encendieran dos teclas laterales. A la tasa que la
paloma pica la tecla iluminada, la computadora generaba pseudo picotazos
que tenían también una probabilidad de 0.05 de apagar la luz de la tecla
central y encender las luces de dos teclas laterales. La tarea para las
palomas era discriminar si el apagado de la tecla central era
consecuencia de uno de sus picotazos o de los producidos por la
computadora, es decir los apagados independientes de su respuesta. Si el
apagado dependía de la respuesta de la paloma, entonces la respuesta de
picar la tecla derecha permitía acceso a la comida; por otro lado, si el
apagado era independiente de su respuesta, entonces el picar la tecla
izquierda era la respuesta que producía acceso a la comida. Los errores
tenían como consecuencia el apagado de todas las luces por un breve
periodo de tiempo. La forma en la cual la paloma informaba sobre su
juicio era por sus respuestas a las teclas laterales.

Como veremos en la práctica sobre ``teoría de detección de señales'',
cuando el organismo identifica correctamente al cambio en la tecla que
es dependiente de su respuesta le llamamos un ``hit''; por el contrario,
cuando el organismo identifica al cambio en la tecla como dependiente de
su respuesta cuando en realidad era el resultado de una pseudo
respuesta, le llamamos una ``falsa alarma''. En la práctica sobre teoría
de la detección de señales (poner link), veremos que las respuestas de
los animales no dependen exclusivamente de su capacidad para detectar
causalidad. En este contexto es que emerge la siguiente pregunta:
¿Cuando los organismos muestran alguna respuesta ``supersticiosa'', esta
se debe a una falla del mecanismo de discriminación o a las ganancias o
costos ligados a emitir una respuesta causalmente errónea? Ponderen que
harían ustedes si un hit produjera \$10, y una falsa alarma les restara
un peso. Ahora consideren que harían si los hits les otorgaran \$1,000 y
las falsas alarmas siguieran teniendo un costo de un peso. Incrementen
ahora la ganancia para los hits a \$10,000. Seguramente, conforme la
ganancia para los hits fuese incrementando, su estrategia se iría
acercando a responder con mayor frecuencia que ustedes produjeron el
cambio, aunque la probabilidad real de que esta relación causal sea
verdadera se mantiene inalterada. En el experimento, Killeen varió la
cantidad de comida que las palomas recibían después de un hit y encontró
que las palomas se comportan justo como lo haríamos nosotros.

¿Qué papel juega el tiempo transcurrido entre una respuesta y el
encuentro con un refuerzo independiente? ¿Si ha transcurrido un tiempo
largo entre una respuesta del organismo y un evento accidental, este
todavía le asignará poder causal a su respuesta para dar cuenta de la
ocurrencia del evento? En el mismo experimento, Killeen se preguntó
acerca del efecto del tiempo transcurrido entre una respuesta y el
apagado de la luz generado por la computadora sobre la probabilidad de
una falsa alarma. Encontró que las falsas alarmas se reducen conforme
ese tiempo se mueve de 0.20 seg. a 1.0 seg. Ver fig.~Es decir, a mayor
distancia entre la respuesta y el suceso accidental, existe una menor
probabilidad de que el organismo le asigne poder causal a su respuesta
para explicar la ocurrencia del evento accidental.

\subsubsection{Conclusiones}\label{conclusiones}

De los experimentos presentados podemos alcanzar las siguientes
conclusiones:

\begin{enumerate}
\def\labelenumi{\arabic{enumi}.}
\item
  La evolución ha seleccionado mecanismos de aprendizaje que consisten
  en buscar los mejores predictores de los sucesos biológicamente
  importantes (SBI).
\item
  Existen sesgos para reducir el tamaño del espacio de
  estímulos/respuestas candidatos predictores de SBI.
\item
  Existen sesgos como la relevancia biológica, la novedad y la
  contigüidad.
\item
  La contigüidad es un factor que influye la selección de
  estímulos/respuestas candidato, pero no es una condición ni necesaria
  ni suficiente para el aprendizaje.
\item
  En los casos en los que se considera a compuestos de estímulos como
  conjuntos de elementos, parece haber competencia entre estos por la
  asignación de crédito para la predicción del SBI.
\item
  El crédito asignado previamente a un elemento del compuesto le resta
  (bloquea) la posibilidad de asignar crédito a otro elemento. Lo
  anterior significa que no todos los elementos contiguos al SBI
  necesariamente son considerados como predictores del mismo.
\item
  La contigüidad estricta entre respuesta y SBI tampoco es necesaria
  para el aprendizaje de respuestas. Las palomas y las ratas pueden
  aprender una respuesta aún con demoras de 32 segundos.
\item
  La mera presentación del refuerzo, independiente de la respuesta, no
  es suficiente para generar el aprendizaje de una respuesta.
\item
  La contigüidad estricta no es necesaria para el aprendizaje de
  respuestas: pero mientras más cercano esté el refuerzo de una
  respuesta, más fácil es su adquisición.
\item
  Aún en protocolos en los cuales no es posible la contigüidad
  accidental entre respuesta y SBI, los animales adquieren la respuesta
  que genera reforzadores demorados en el tiempo.
\item
  Las palomas pueden discriminar entre consecuencias dependientes e
  independientes de su respuesta.
\item
  El que un organismo juzgue a una consecuencia como dependiente de su
  respuesta varía en función de la ganancia y el costo asociado a esos
  juicios.
\end{enumerate}

\bookmarksetup{startatroot}

\chapter{Correlación, Tiempo y
Contingencia}\label{correlaciuxf3n-tiempo-y-contingencia}

En mi juventud, los periódicos más amarillistas de la Ciudad de México
tenían titulares de ocho columnas, del tipo ``mariguano ataca con un
cuchillo a su vecino''. Titulares de este tipo eran utilizados por los
noticieros para justificar la prohibición del consumo de la mariguana.
Independientemente del debate acerca de su legalización, esos titulares
tienen un importante problema que dificulta atribuir al consumo de la
droga el ataque perpetuado. El error radica en asignar el crédito del
ataque a la mariguana, observando únicamente lo que sucede después del
consumo de la droga. Los periódicos y los noticieros debían preguntarse
adicionalmente cuántos ataques a vecinos ocurren cuando el atacante
\textbf{No} está bajo la influencia de la droga. Si el número de ataques
a vecinos fuese similar cuando el atacante consumió la droga y cuando
este no lo hizo, consideraríamos que el consumo de esta sustancia no se
encuentra correlacionado con los ataques. De la misma manera, nos
planteamos las dos preguntas sutilmente diferentes pero relacionadas de:
¿cuántos vecinos \textbf{No} fueron atacados cuando se encontraron con
una persona que \textbf{Sí} había consumido la droga? y ¿cuántos vecinos
\textbf{No} fueron atacados cuando se encontraron con una persona que
\textbf{No} la había consumido? Si el número de ataques y de no ataques
a vecinos en encuentros con otras personas es similar,
independientemente de si las personas han consumido la droga o no,
consideraríamos que su consumo no está correlacionado con los ataques.

En este capítulo, nos preguntaremos si las palomas y las ratas son
sensibles únicamente a lo que ocurre después de un estímulo o de una
respuesta o si para la asignación de crédito, estos organismos también
contemplan lo que ocurre cuando el estímulo o la respuesta en cuestión
no se encuentran presentes. En otras palabras, ahondaremos con mayor
profundidad sobre la pregunta de si la contigüidad entre un estímulo (o
una respuesta) y un reforzador es un elemento suficiente y necesario
para la asignación de crédito.

Los protocolos experimentales presentados en los dos capítulos
anteriores tienen en común el variar lo que ocurre después de una
estímulo o de una respuesta, ya sea con un refuerzo inmediato o
demorado. En 1968, Rescorla introdujo un protocolo experimental que
permite manipular lo que ocurre en la presencia y en la ausencia de un
estímulo o una respuesta. Al protocolo le llamó \emph{verdaderamente
aleatorio}. En lugar de presentar un refuerzo al final del estímulo
condicionado (EC), Rescorla varió la probabilidad de un refuerzo durante
la presencia y durante la ausencia del estímulo condicionado (ver
figura). En su experimento, Rescorla presentó un estímulo condicionado
de una duración de 20 seg. con un intervalo de 2 min. entre cada
presentación del EC: a este último periodo se le conoce como el
intervalo entre ensayos. Durante ambos periodos de tiempo, la
presentación del refuerzo se determina con cierta probabilidad para cada
segundo. Las diferencias entre las probabilidades en los dos periodos de
tiempo pueden manipularse, de tal forma que la probabilidad de un
refuerzo durante la presencia del EC sea mayor o menor que la
probabilidad durante el intervalo entre ensayos. En el primer caso,
diremos que el EC y el refuerzo están correlacionados positivamente, en
el segundo caso diremos que están correlacionados negativamente. Si las
probabilidades en ambas duraciones son iguales diremos que ambos
elementos no se encuentran correlacionados. La figura x ilustra el
espacio de posibles correlaciones. En el eje de las X se presenta la
probabilidad de refuerzo durante el intervalo entre ensayos, mientras en
el eje de las Y se presenta la probabilidad de refuerzo durante el
estímulo condicionado. La línea diagonal representa la falta de
correlación entre el EC y el refuerzo. El espacio arriba de la diagonal
representa correlaciones positivas y el espacio abajo de la diagonal
representa correlaciones negativas. Los puntos cercanos al 1 y al cero,
representan las correlaciones más fuertes: en un caso, los refuerzos
ocurren exclusivamente durante el EC, en el otro, ocurren casi
exclusivamente durante el intervalo entre ensayos.

Usando diferentes medidas de aprendizaje, múltiples experimentos han
encontrado que en la condición \emph{no correlacionada} los animales no
le asignan crédito al EC. En cambio, cuando la correlación es positiva,
los animales aprenden que el EC predice al refuerzo, y por el contrario,
cuando la correlación es negativa, los animales aprenden que el EC
predice la ausencia del refuerzo. Esos resultados añaden evidencia a la
afirmación de que la contigüidad entre el EC y el refuerzo no es una
condición suficiente para la asignación de crédito. Cuando es igualmente
probable que el refuerzo aparezca en la presencia del EC como en su
ausencia, en el entorno del organismo no hay una relación causal entre
el EC y el refuerzo. Intuitivamente, en la condición no correlacionada,
el EC no proporciona información alguna acerca de la ocurrencia de los
refuerzos. Podríamos eliminar del protocolo experimental el EC y no se
alteraría la expectativa del organismo acerca de la ocurrencia de los
refuerzos. En otro capítulo, veremos que formalizar la noción intuitiva
de información ilumina muchos de los hallazgos acerca de la asignación
de crédito.

Los organismos no tan solo son sensibles a la correlación entre
estímulos y refuerzos, a continuación veremos que también son sensibles
a la correlación entre respuestas y refuerzos. Para evaluar si la
contigüidad entre la respuesta y el refuerzo es suficiente para asignar
crédito o si es necesario que exista una correlación entre estos
elementos, Hammond corrió un experimento con una estructura similar al
de Rescorla, manipulando la probabilidad de un refuerzo dada la
ocurrencia y ausencia de una respuesta. Para lograr igualar la
oportunidad de una respuesta y de una no respuesta, Hammond partió el
tiempo de la sesión experimental en segmentos de un segundo de duración.
En cada uno de esos segundos el animal puede o no emitir una respuesta;
asimismo, este puede obtener o no, con cierta probabilidad, un refuerzo.
La probabilidad de obtener un refuerzo en cada segundo dependía de la
presencia o ausencia de una respuesta por parte del organismo. En todas
las condiciones del experimento, Hammond mantuvo constante la
probabilidad de un refuerzo dada la ocurrencia de una respuesta en el
intervalo de un segundo, y varió su probabilidad en la ausencia de la
respuesta. La probabilidad del refuerzo dada la respuesta fue de 0.05,
mientras que la probabilidad de un refuerzo dada la no respuesta fue de
cero o de 0.05. La figura x muestra que las ratas no responden en las
condiciones experimentales en las cuales las dos probabilidades de
refuerzo eran iguales.

\section{Tiempo}\label{tiempo}

Un problema de dirigir nuestra atención exclusivamente al momento en el
que ocurre el estímulo condicionado (EC) o la respuesta, es que solo
atendemos a los refuerzos contiguos al EC; en ese sentido, también
emerge el problema de ignorar el tiempo entre presentaciones del EC, lo
que llamamos el intervalo entre ensayos (TEE). Si la contigüidad
estímulo-respuesta fuera la única variable importante para el
aprendizaje de comportamientos, entonces manipular la duración del EC
relativa a la duración del intervalo entre ensayos no tendría ningún
efecto sobre el aprendizaje.

Un escenario hipotético nos hace dudar de la conclusión anterior.
Comparemos dos fábricas, en ambas, cada cuatro horas hay 10 minutos de
descanso. Pero en una de las fábricas, el período de descanso es
precedido por una señal que dura 3 horas y 45 minutos, mientras que en
la otra, el descanso es precedido por una señal que dura 10 minutos.
Pregúntense si ambas señales les serían igualmente informativas, si les
prestarían igual atención a ambas o si una de ellas les permitiría
anticipar adaptativamente la ocurrencia del descanso. Intuitivamente,
este caso hipotético nos hace pensar que los escenarios con EC con una
duración muy larga respecto a la duración de los intervalos entre
ensayos (TEE) inducen un menor aprendizaje que los escenarios con EC que
tienen una duración más corta respecto a la duración de los TEE.

Gibbon et al.~llevaron a cabo justo ese protocolo con ratas para evaluar
la importancia de las duraciones del intervalo entre ensayos y del
estímulo condicionado. En el experimento, se manipularon dos
condiciones: en la primera, se incrementó la duración del estímulo
condicionado, manteniendo constante el tiempo del intervalo entre
ensayos; y en la segunda condición, se incrementaron ambas duraciones
proporcionalmente (por ejemplo, de 4 a 8 y de 48 a 96 segundos,
respectivamente). La medida del aprendizaje fue el número de refuerzos
necesarios para mostrar aprendizaje. La siguiente figura muestra los
resultados. La línea roja muestra que el número de refuerzos requeridos
para el aprendizaje incrementó como función del aumento en la duración
del EC, manteniendo constante la duración del intervalo entre ensayos:
bajo esta condición, las señales fueron ``menos informativas'' para los
organismos, por lo que el aprendizaje resultó ``más difícil''. Por el
contrario, incrementar las dos duraciones, manteniendo constante su
razón, no tuvo un efecto sobre el número de refuerzos necesarios para el
aprendizaje.

Lo anterior indica que los organismos son sensibles a la razón TEE/TEC:
si esta razón rebasa un valor, el animal aprende acerca de la
importancia del EC. Una forma de entenderlo es suponer que el EC reduce
la incertidumbre acerca del momento de ocurrencia de un refuerzo:
mientras la razón TEE/TEC sea más grande, mayor es la reducción en la
incertidumbre de la entrega. En otras palabras, la asignación de crédito
a un estímulo o respuesta depende de que el estímulo condicionado sea
breve relativo al intervalo entre las presentaciones del refuerzo. En
nuestro ejemplo hipotético, el obrero de la fábrica aprenderá acerca de
la señal de 10 minutos que ocurre cada 4 horas, y no de la señal de 3
horas y media de duración.

Conclusiones

Adicionalmente a las limitaciones ya señaladas sobre el papel de la
contigüidad en la selección de estímulos/respuestas candidato, cabe
agregar las siguientes observaciones:

\begin{enumerate}
\def\labelenumi{\arabic{enumi}.}
\item
  La correlación en el tiempo entre EC y EI es un factor importante en
  el aprendizaje.
\item
  Uno de los criterios para identificar un buen predictor de SBI es su
  duración relativa a la duración del intervalo entre ensayos: en
  particular, la razón TEE/TEC.
\end{enumerate}

\bookmarksetup{startatroot}

\chapter{Modelo de Aprendizaje por
Refuerzo}\label{modelo-de-aprendizaje-por-refuerzo}

Como un resultado de la selección natural, los agentes biológicos son
sistemas con mecanismos que les permiten detectar, predecir y controlar
los sucesos biológicamente importantes (SBI). Recordemos que los SBI son
sucesos con valor hedónico, también conocidos como refuerzos, los cuales
incrementan el éxito reproductivo diferencial de los organismos. Para
ejecutar estas funciones de detección, predicción y control de SBI los
organismos necesitan hacer contacto con la estructura causal del
entorno: estructura a la cual designamos bajo el término de
\emph{propiedades estadísticas del entorno}.

Una parte de la estructura estadística de los SBI consiste de estímulos
que se despliegan en el tiempo y en el espacio, los cuales en algunas
ocasiones aparecen solos, mientras que en otras instancias aparecen
seguidos (contiguos) de un refuerzo. Hay tiempos sin nubes y otros con
nubes, estos últimos pueden ser seguidos o no de lluvia. La tarea para
el organismo es determinar si existe una relación causal entre las nubes
y la lluvia. Una segunda parte de la estructura estadística del entorno
consiste del hecho de que, con relación al despliegue de respuestas de
los organismos en el tiempo y en el espacio, algunas ocasiones estas
respuestas van seguidas de un refuerzo y en otras ocasiones no. Hay
veces en las que ustedes le dicen ``hola'' a una persona y otras en las
que no. Después del ``hola'' ustedes pueden recibir, o no, otro ``hola''
de regreso. Los protocolos de condicionamiento clásico e instrumental
definen, como una primera aproximación, las estructuras causales más
simples que se estudiaron en buena parte del siglo XX. En los dos
capítulos anteriores, resumimos la literatura empírica acerca del papel
de la contigüidad en la asignación de crédito a estímulos y respuestas
que permiten \emph{predecir y controlar} los refuerzos. Presentamos
evidencia que muestra que aunque la contigüidad es importante, esta no
es ni necesaria, ni suficiente para la asignación de crédito.

Para dar cuenta de los resultados empíricos presentados en los capítulos
anteriores, en la segunda mitad del siglo XX se consolidó un modelo
matemático, conocido como \emph{Aprendizaje por Refuerzo}, junto con
varias modificaciones del mismo. El propósito de este capítulo es
presentar estos modelos y su desarrollo como respuesta a la evidencia
sobre el papel de la contigüidad.

.

\subsection{Curvas de Aprendizaje}\label{curvas-de-aprendizaje}

El aprendizaje es un proceso dinámico, que describe los cambios en el
comportamiento como una función de la experiencia de los organismos.
Desde finales del siglo XIX se han obtenido ``curvas de Aprendizaje''
que describen cambios en alguna medida de ejecución de los organismos
como una función de las ocasiones en las que los estímulos que encaran o
las respuestas que despliegan van seguidos de un refuerzo.

Como ilustración, presentamos dos ejemplos, el primero es la curva de
adquisición y extinción de la frecuencia del reflejo de parpadeo, en un
protocolo de condicionamiento clásico con humanos. Las curvas
representan diferentes intensidades del soplo al ojo. (Parssey, 1948)

IMAGEN

Un segundo ejemplo es la curva de adquisición de la velocidad de tocar
una palanca en un protocolo de condicionamiento instrumental con ratas
(Ramond, 1954). Las dos curvas representan los datos obtenidos con
diferentes niveles de privación.

IMAGEN

La siguiente curva de adquisición idealizada, captura los datos de
adquisición presentados en las dos figuras anteriores. Podemos ver que
es una curva negativamente acelerada de ganancias decrecientes, esto es,
el impacto de que un refuerzo siga a una respuesta se va reduciendo
conforme se acumulan las ocasiones en las que una respuesta va seguida
de un refuerzo.

IMAGEN

Nota. Una alternativa teórica supone que el aprendizaje no es un proceso
gradual, sino uno de cambio abrupto. Si este fuese el caso, las curvas
de adquisición de crecimiento gradual decreciente podrían ser un
artefacto de promediar la ejecución de animales con cambios \emph{no}
continuos en el aprendizaje. Considere el caso de un grupo de animales
cuya ejecución es promediada. Uno de los animales empieza a responder al
ensayo 10, otro al 20, otro al 30, otro al 40 y así hasta un animal que
responde al ensayo 80: al promediar estos datos, la curva de aprendizaje
aparecerá como una curva continua. Por esta razón, Skinner, Estes,
Spence y más recientemente Gallistel argumentan en favor de analizar los
datos de sujetos individuales y, preferentemente, registros acumulativos
en lugar de los datos promediados. En ese mismo sentido, los modelos que
presentaremos a continuación resultan válidos cuando los datos
analizados provienen de sujetos individuales.

Los modelos de refuerzo modelan la forma de las curvas de adquisición
obtenidas empíricamente. El modelo de refuerzo más general es un sistema
dinámico que describe los cambios en el valor de un estímulo y/o
respuesta a lo largo del tiempo, como una función del número de
ocasiones en las que un estímulo o una respuesta van seguidas de un
refuerzo.

Pasen al simulador de ecuaciones en diferencia para entender los modelos
dinámicos discretos.

\section{Modelo de Refuerzo}\label{modelo-de-refuerzo}

Los modelos de refuerzo le asignan un número a cada estímulo y/o
respuesta: esta magnitud representa la calidad del estímulo/respuesta
como predictor de un refuerzo. A lo largo del siglo XX, a esta magnitud
se le conoció como fuerza del reflejo, fuerza del hábito, fuerza
asociativa del estímulo y fuerza de la respuesta. Para nuestros
propósitos el número refleja el \emph{valor} predictivo de un estímulo o
de una respuesta y simplemente le llamaremos el valor V del estímulo i,
o el valor Q de la respuesta i.

El modelo de refuerzo propone que después de cada presentación de un
estímulo o la emisión de una respuesta, su valor se actualiza como una
función de si este va seguido o no de un refuerzo. Es importante señalar
que el modelo asume que la actualización del valor de un estímulo o una
respuesta solo ocurre en las ocasiones en las que estos se presentan. El
valor predictivo de un tono solo se actualiza cuando el tono se presenta
y no en su ausencia. Esto implica que el mero paso del tiempo sin el
tono o la presentación de otros estímulos no alteran el valor del tono.
Por esta razón, a estos modelos se les llama también modelos basados en
ensayos. Esto es, la variable importante que determina el valor
predictivo (la asignación de crédito) es el número de ocasiones en las
que un estímulo o una respuesta son seguidos de un reforzador. Mientras
mayor es este número, mayor será el valor adquirido por el estímulo o la
respuesta. De forma complementaria, el valor de estos estímulos o
respuestas decrementa cuando estos se presentan sin ser seguidos por el
reforzador.

En 1950, Bush y Mosteller propusieron una versión matemática del modelo
de aprendizaje por refuerzo esbozado en el párrafo anterior. Esta
versión sigue siendo la base que sustenta varios de los modelos más
recientes, tanto en la psicología como en el aprendizaje de máquinas.

La estructura matemática de los modelos de refuerzo tiene diferentes
interpretaciones teóricas. Nosotros consideraremos dos:

\begin{enumerate}
\def\labelenumi{\arabic{enumi}.}
\tightlist
\item
  Un proceso de carga - descarga y
\item
  Un proceso de reducción del error
\end{enumerate}

\subsection{Proceso de carga -
descarga}\label{proceso-de-carga---descarga}

Los modelos de refuerzo son una propuesta de solución computacional a la
asignación de crédito. La solución incluye dos pasos: el primero es la
reducción del tamaño inicial del espacio de candidatos para incluir
únicamente sucesos que son contiguos, similares, novedosos y
evolutivamente relevantes con relación a los SBI. El segundo es un
mecanismo que permita ir reduciendo, a través de la experiencia, el
espacio de candidatos hasta terminar con uno solo. El modelo de refuerzo
canónico combina un algoritmo de ascenso de colina con el sesgo de
contigüidad. Bush y Mosteller (1951) formalizaron esta clase de modelos
que, en diferentes variantes, han dominado la literatura teórica y
experimental en el estudio del aprendizaje a partir de la década de los
70s del siglo pasado.

Como ya se dijo antes, la forma más literal de interpretar el modelo de
refuerzo es como un proceso en el cual: cada refuerzo incrementa (carga,
fortalece) el valor del estímulo / respuesta que le antecede y cada
ocurrencia del estímulo / respuesta sin ser acompañado de un refuerzo
decrementa (descarga) su valor. Este es un proceso en el que la variable
\(V\) se actualiza con cada ocurrencia del estímulo / respuesta como una
función que varía dependiendo de si el estímulo/respuesta va seguida o
no de un refuerzo. La carga de la batería de su celular es un ejemplo
muy cercano a su vida cotidiana que ejemplifica el proceso descrito.
Para que su celular funcione, ustedes tienen que conectarlo a una toma
de corriente. Mientras está conectado, la carga de la batería se va
actualizando hasta llegar a un punto máximo. Al usarlo sin tenerlo
conectado, la batería se va descargando como una función del uso del
celular sin una carga adicional.

Veamos cómo se aplica este modelo en el caso de un protocolo estándar de
condicionamiento clásico: en este contexto, se observa un
estímulo/respuesta candidato a la asignación de crédito y a cada
presentación de él se le conoce como un ensayo. Cada ocurrencia de un
ensayo puede estar acompañado o no de un SBI. Consideremos que \(Vx\)
represente el valor predictivo de un estímulo \(x\) . Nos interesa la
dinámica del cambio en \(Vx\) conforme un organismo experimenta ensayos
en los que un estímulo condicionado \(x\) va seguido de un refuerzo.
Para facilitar la aplicación del modelo, supondremos que el tiempo es
discreto y el subíndice \(t\) representa el momento en que se presenta
el estímulo \(x\). La variable \(Vx_t+1\) es el valor actualizado del
estímulo \(x\) en el siguiente ensayo \(t+1\). La variable \(R\)
representa si se presentó o no un refuerzo después del estímulo \(x\) .
\(R\) puede tener solo dos valores, uno si el refuerzo se presenta
después del estímulo \(x\) o cero si el estímulo \(x\) se presenta sin
ser seguido por el refuerzo.

Vamos a asumir que \(Vx_t+1\) depende sólo de dos factores: 1. su valor
acumulado hasta el ensayo inmediatamente anterior \(Vx_t\) 2. el valor
de \(R_t\) en el ensayo actual

\[
Vx_{t+1} = Vx_t + R
\]

\begin{enumerate}
\def\labelenumi{\arabic{enumi}.}
\setcounter{enumi}{2}
\tightlist
\item
  La expresión anterior supone que el efecto del valor de V en el ensayo
  anterior tiene el mismo peso que el reforzador presentado o no en el
  momento actual. Nosotros deseamos que la ecuación capture la
  importancia relativa de las dos variables. Para ello, supondremos que
  el impacto de esas dos variables es una suma ponderada, donde el
  parámetro \(a\) representa el peso de ponderación asignado a cada uno
  de los dos factores. Si el valor de \(a\) esta entre cero y uno, los
  parámetros asociados con cada factor serán \(a\) y \((1-a)\).
\end{enumerate}

\[
Vx_{t+1} = (1-a)Vx_t + aR_t
\]

donde: \(0 < a < 1\).

La ecuación anterior es una ecuación recurrente, en la que en cada
iteración (presentación del estímulo \(x\)) hay siempre un Vx viejo y un
Vx nuevo. La ecuación describe la actualización de \(Vx\) de ensayo a
ensayo, como una función del valor de \(Vx\) acumulado hasta el ensayo
anterior y la presentación o ausencia del SBI. Vx\_t es la integración,
el acumulado, de todas las experiencias previas con el refuerzo. En cada
ensayo, la Vx nueva del ensayo anterior se convierte en la Vx vieja del
presente ensayo, contribuyendo a generar una nueva Vx. Pase al simulador
x para revisar ecuaciones recurrentes.

El parámetro \(a\), que multiplica al refuerzo, determina la velocidad
del aprendizaje: mientras mayor sea su valor, más rápido será el
aprendizaje. Una forma de entender el papel de \(a\), es considerarlo
como el parámetro que especifica la importancia de la experiencia
acumulada hasta el momento (el valor del pasado), relativa a la
ocurrencia o no de un refuerzo (el valor del presente). Valores cercanos
a cero sugieren que la experiencia acumulada es más importante que una
nueva experiencia con un refuerzo, resultando en poco aprendizaje,
mientras que valores cercanos a uno sugieren que la presentación del
refuerzo es más importante que la experiencia acumulada hasta ese
momento, resultando en un rápido aprendizaje. Consideren una interacción
de larga duración con una amistad que ha resultado en un valor alto para
ustedes asociado con esa relación. Nos podemos preguntar cuál es el
impacto de que una mañana esa amistad no los salude. Si el parámetro
\(a\) fuese cercano a cero, el no saludo no modificaría sustancialmente
el valor V de la amistad, mientras que si el valor de \(a\) fuese
cercano a uno, a pesar de los años de experiencias positivas con esa
amistad, su impacto en el valor V sería más significativo.

Las siguientes figuras muestran los resultados, ensayo a ensayo, de una
simulación con diferentes valores del parámetro \(a\).

IMAGENES

\subsection{Reducción del error de predicción como motor del
aprendizaje}\label{reducciuxf3n-del-error-de-predicciuxf3n-como-motor-del-aprendizaje}

El reacomodo de los términos de la ecuación de carga - descarga permite
una interpretación alternativa del modelo de aprendizaje por refuerzo:
esta vez en términos de un mecanismo de reducción en el error de
predicción. Estos modelos representan hoy la versión dominante en la
psicología del aprendizaje y las neurociencias.

Arreglando los términos de la ecuación del integrador con fuga:

\[
V_{t+1} = (1-a)V_t + aR_t
\]

\[
V_{t+1} = V_t - aV_t + aR_t
\]

\[
V_{t+1} = V_t + a (R_t -V_t)
\]

Restándole \(V_t\) a ambos lados de la ecuación anterior, dejando que
\(\Delta Vx = Vx_t+1 – Vx_t\) entonces el cambio de ensayo a ensayo es
descrito, agrupando términos:

\[
\Delta Vx = V_t + a (R_t -V_t) - V_t
\]

\[
\Delta Vx = a(R_t – Vx_t)
\]

Esta segunda forma de la ecuación, enfatiza la magnitud del cambio
momento a momento, en lugar del valor del estímulo momento a momento.

En cualquiera de las dos formas de presentar la ecuación del integrador,
el valor V es una función de la diferencia entre el refuerzo obtenido y
el valor del estímulo en el tiempo t. A esta diferencia dentro del
paréntesis se le conoce como el error de predicción: la diferencia entre
la R que se obtiene y lo que se esperaba obtener, V. Cuando esta
diferencia es igual a cero, no habrá cambios en el valor de V. Es por
esta forma de la ecuación que estamos llamando a V\_t el valor
predictivo de la respuesta. El parámetro a, entre 0 y 1, sigue
representando la velocidad del aprendizaje, en este caso, la importancia
del error de predicción.

En la literatura contemporánea, a la ecuación anterior se le conoce como
regla delta.

\[
V_{t+1} = V_t + a \Delta Vx
\]

donde \(%
\)delta = (lambda\_t - Vx\_t)\(%
\) es el error de predicción, es decir, el motor del aprendizaje. Bajo
el formato que enfatiza la magnitud del cambio, delta se incorpora a la
ecuación de la siguiente forma:

\(%
\)∆Vx = a(delta)\(%
\)

\bookmarksetup{startatroot}

\chapter{El Modelo de Rescorla y
Wagner}\label{el-modelo-de-rescorla-y-wagner}

En el capítulo anterior, vimos que el modelo de aprendizaje por
refuerzo, conocido también como modelo de \emph{aprendizaje de error de
predicción} captura razonablemente bien la adquisición de valor
predictivo cuando un estímulo o respuesta van seguidos de un refuerzo.
En este capítulo veremos que este modelo \emph{no} puede dar cuenta de
los resultados presentados en el capítulo sobre asignación de crédito,
los cuales ilustran la importancia de una serie de correlaciones:
aquellas que sostiene el EC o la respuesta con la aparición del
refuerzo; aquellas que sostiene el EC con otros estímulos presentes
(elementos del contexto); y aquellas que existen previamente entre otros
estímulos distintos y el refuerzo actual. Originalmente, estos últimos
resultados indujeron interpretaciones que enfatizaban que la asignación
de crédito a un estímulo o respuesta dependía de que estos fueran
seguidos de un refuerzo que era \emph{sorpresivo, inesperado,
informativo, o que atraía la atención. } En 1972, Rescorla y Wagner
presentaron un modelo que daba cuenta de los resultados que muestran que
la mera contigüidad no es un factor necesario ni suficiente para la
asignación de crédito. El modelo es una extensión del principio de la
reducción de error, que captura la intuición acerca del papel de la
sorpresa como un modelo matemático: todo ello sin hacer referencia a
procesos atencionales que se suponían difíciles de evaluar con sujetos
no humanos. Este modelo sigue siendo hasta la fecha el motor de la
investigación en aprendizaje.

\subsection{Modelo de Rescorla y
Wagner}\label{modelo-de-rescorla-y-wagner}

El modelo de Rescorla y Wagner incluye dos grandes componentes. El
primer componente es el \emph{Modelo de Refuerzo} de Bush y Mosteller,
el cual hemos visto que establece la reducción en el error de predicción
como el motor del aprendizaje. El segundo componente es un modelo de la
forma en la que un organismo percibe estímulos compuestos. En
particular, este modelo supone que los organismos perciben a los
estímulos, por ejemplo un rostro, como un conjunto de elementos
separables: en este caso, un rostro se percibe en términos de ojos,
nariz, labios, entre otros. El modelo asume que todos estos elementos
compiten entre ellos por la asignación de crédito.

El modelo de aprendizaje utilizado por Rescorla y Wagner es una variante
del modelo de la reducción en el error de la predicción (también
conocido como la regla delta).

\[Vx_{t+1}= Vx_t + a (R_t -Vx_t)\] Donde 0 \textless{} a \textless1

Para entender el segundo componente del modelo de Rescorla y Wagner,
consideremos por un momento las características del entorno modelado.
Hasta antes de los años 60s del siglo pasado, los investigadores
limitaban sus experimentos a protocolos en los que se presentaba un solo
estímulo condicionado. Sin embargo, los entornos reales no consisten de
elementos que aparecen aisladamente y cuyo único aspecto complejo es la
variabilidad en su distancia temporal respecto al SBI. Por el contrario,
los organismos encaran entornos en los que múltiples estímulos se
presentan simultáneamente y en ocasiones de manera contigua con los
refuerzos. Una comida que nos enferma o que nos produce un gran placer
es en sí misma un compuesto de múltiples estímulos: el plato en que se
sirve, el mantel bajo el plato, cómo se ve, su aroma, la música que se
está escuchando, la persona que la sirve. Más aún, cada uno de nosotros
tenemos experiencias diferenciadas con cada uno de estos elementos por
separado, correlacionados con otros o con el mismo reforzador. Hemos
comido en ese mantel otras comidas, con platos y aromas diferentes. El
modelo de Rescorla y Wagner describe el algoritmo, la regla por la cual
se le asigna crédito a cada elemento de la experiencia con una comida.
En ese sentido, el modelo le da respuesta a la pregunta: ¿Cómo puede un
organismo extraer relaciones de ``causalidad'' en ésta red de diversas
experiencias? En resumen, el modelo de Rescorla y Wagner captura los
principios que describen la asignación de crédito a los distintos
elementos de un estímulo compuesto que es seguido por un reforzador. Al
mismo tiempo, el modelo especifica el efecto que juega la experiencia
previa del agente con cada uno de los elementos por separado dentro de
la asignación de crédito.

\subsection{Supuestos del Modelo de Rescorla y
Wagner}\label{supuestos-del-modelo-de-rescorla-y-wagner}

\subsubsection{El supuesto de la separabilidad de los
estímulos.}\label{el-supuesto-de-la-separabilidad-de-los-estuxedmulos.}

Los estímulos en compuesto están conformados por elementos (estímulos)
separables. Desde esta perspectiva, una cara, por ejemplo, no es un
estímulo integrado, sino un conjunto de elementos (ojos, boca, orejas,
nariz, etc.).

\subsubsection{El supuesto del valor predictivo de los
elementos}\label{el-supuesto-del-valor-predictivo-de-los-elementos}

Cada elemento de un compuesto, sea un estímulo o una respuesta, tienen
un número ligado a ellos; a este número le llamamos Valor. El valor
puede tomar números positivos pero también negativos. Cuando el valor es
positivo predice la ocurrencia de un refuerzo, cuando es negativo
predice su ausencia. Por esta razón, a dicho número también se le conoce
como el valor predictivo del estímulo. El valor (V) se actualiza en cada
ocasión que se presenta el estímulo o respuesta (EC) y el cambio en la
magnitud del mismo depende de si el EC se presenta acompañado o no de un
suceso biológicamente importante (R). La relación entre el valor y
alguna medida de comportamiento es únicamente ordinal. Las diferencias
en valor sólo predicen diferencias en el ordenamiento de alguna medida
del comportamiento. En otras palabras, un elemento de un estímulo
compuesto (por ejemplo, la nariz en los rostros) con un valor predictivo
V de 1 no induce el doble de respuestas en un agente con relación a otro
elemento del estímulo compuesto que tenga un valor de 0.5 (por ejemplo,
el vello en los rostros): lo único que nos señalan estos valores
numéricos es que el agente le está asignando mayor crédito por la
ocurrencia del refuerzo a la nariz con relación al vello de los sujetos,
y que el agente responderá más ante estímulos que contengan narices que
ante estímulos que contengan vello (sin especificar cuantitativa ni
precisamente esta diferencia de respuestas).

\subsubsection{La regla de integración del valor de los
elementos}\label{la-regla-de-integraciuxf3n-del-valor-de-los-elementos}

El modelo computa por separado, para cada uno de los elementos de un
compuesto, su valor predictivo V, y el valor predictivo del compuesto es
la suma de los valores predictivos de cada uno de sus elementos. Si el
compuesto incluye dos estímulos A y B, se computan por separado VA y VB.
La fuerza de la predicción del compuesto es la suma de los Vs es, en
nuestro caso:

Vtotal = VA+ VB.

\subsubsection{La regla de la actualización del valor predictivo de los
elementos.}\label{la-regla-de-la-actualizaciuxf3n-del-valor-predictivo-de-los-elementos.}

La ecuación de Rescorla y Wagner mantiene el supuesto de que la
asignación de crédito a cada uno de los elementos de un compuesto es una
función de la discrepancia entre lo que se obtiene y lo que se espera
obtener. La contribución de Rescorla y Wagner es suponer que lo que se
espera obtener dada la presentación de un compuesto es el resultado de
la suma del valor predictivo de todos los elementos presentes
simultáneamente (V\_total).\_

Vx\_t+1 = Vx\_t + a(R-Vtotal\_t)\_

Recuerden que en nuestro protocolo, R es un valor binario que representa
la presentación (R=1) o no (R=0) de un refuerzo. Como en la ecuación de
Bush y Mosteler, \emph{a} es un parámetro de aprendizaje que determina
la importancia del error de predicción.

La ecuación especifica la reducción del error de predicción como motor
del aprendizaje y, como puede verse en el simulador, este modelo produce
curvas de aprendizaje de ganancias decrecientes, en las cuales el cambio
en V es cada vez más pequeño conforme el error de predicción se reduce.
La parte novedosa de la ecuación consiste en tomar como predicción la
suma del valor de todos los elementos presentes: lo cual nos conduce al
último supuesto del modelo\ldots{}

\subsubsection{Competencia entre los elementos de un
compuesto}\label{competencia-entre-los-elementos-de-un-compuesto}

Los elementos separados compiten entre sí por el valor predictivo del
compuesto que conforman. Recordemos que el valor predictivo total de un
estímulo compuesto es limitado, lo que implica que mientras mayor sea el
valor de uno de los elementos, quedará menos ``valor predictivo'' para
ser distribuido a los demás elementos del compuesto.

\subsubsection{La ecuación de Rescorla y
Wagner}\label{la-ecuaciuxf3n-de-rescorla-y-wagner}

Vx\_\{t+1\} = Vx\_t + alphabeta(R-Vtotal\_t)

En ocasiones la ecuación de Rescorla y Wagner se representa en términos
de los cambios de ensayo a ensayo.

triangle Vx =Vx\_\{t+1\} - Vx\_t

Las preguntas que emergen al considerar este modelo son: primero, ¿de
qué variables depende el parámetro ``a''? y segundo, ¿es ``a'' el único
parámetro que determina la velocidad del aprendizaje? Empíricamente,
podemos considerar dos variables: en primer lugar, la naturaleza del
refuerzo. Por ejemplo, más y mejor comida produce un aprendizaje más
rápido. Una segunda variable es la naturaleza del estímulo predictor. Un
estímulo más intenso o sobresaliente produce curvas de aprendizaje más
aceleradas. La importancia de este segundo elemento -la saliencia del
EC- se representa en la ecuación con un parámetro adicional de
aprendizaje que llamaremos beta, el cual también adquiere valores entre
cero y uno y también se multiplica por el error de predicción para
ponderar su importancia relativa.

triangle Vx = alpha beta (R-Vtotal\_t)

Para ayudar a entender el modelo de Rescorla y Wagner en su aplicación a
la vida cotidiana, consideren el siguiente escenario. Un amigo al que
ustedes visitan con frecuencia consiguió un nuevo perro. A ustedes les
gustaría saber si este es un perro al que se le puede acariciar sin
temor a que este los muerda. El perro es un compuesto de múltiples
elementos: tamaño, hocico, ojos, orejas, tipo de pelo, entre otros. Su
primera respuesta ante ese nuevo perro va a ser el resultado de la suma
de los valores predictivos de los distintos elementos que lo componen,
adquiridos de sus múltiples experiencias con otros perros. Por ejemplo,
imaginemos que en algún momento del pasado se encontraron con un perro
pequeño y chato que nunca trató de morderlos; posteriormente, cuando se
encuentran con el perro de su amigo que comparte el mismo tamaño chico
de aquel perro pero que tiene un hocico largo, su predicción sobre si
este los morderá será la suma de lo que para ustedes predicen, por
separado, su tamaño y su tipo de hocico. En este caso, el tamaño (y no
el tipo de hocico) del perro de su amigo tendrá un valor predictivo en
el sentido de que el animal no les morderá. Por otra parte, si el perro
de su amigo intenta morderlos, el valor de los dos atributos se
actualizará a través del error de predicción. Es decir, si la suma de
los valores predictivos de los elementos del perro de su amigo predijo
en un inicio que este no los mordería, y este efectivamente procede a
morderlos, entonces habrá un error de predicción que actualizará el
valor de cada uno de los elementos que conforman al perro. De esta
forma, el elemento del tamaño chico del perro perderá su valor como un
predictor de una ``no mordida'', mientras que el elemento del ``hocico
largo'' adquirirá valor como un predictor de una ``mordida''.

\subsection{Aplicación del modelo de Rescorla y Wagner al experimento de
ensombrecimiento}\label{aplicaciuxf3n-del-modelo-de-rescorla-y-wagner-al-experimento-de-ensombrecimiento}

Considere el protocolo experimental mostrado en la figura x. Hay tres
grupos, para el grupo G1 en cada ensayo se presenta un compuesto de dos
estímulos (tono y luz) seguidos por el acceso a alimento. Para otros dos
grupos solo se les presenta el tono o la luz, cada uno seguido de
comida. Vamos a suponer que el tono y la luz no son igualmente
sobresalientes (es decir, tienen betas diferentes). Para el grupo G1, el
error de predicción es R menos la suma del valor adquirido en cada
ensayo por los elementos del compuesto, para el cual: Vtotal = Vluz +
VTono. Para los otros dos grupos, el error de predicción es R menos el
valor de cada elemento por separado, VT o VL. En la simulación puede
verse que cuando se tiene un tono ligeramente más sobresaliente que la
luz en el grupo con el estímulo compuesto, ni el tono ni la luz alcanzan
valores cercanos a R. En los grupos en los cuales los dos estímulos se
presentan por separado, ambos estímulos alcanzan valores que se
aproximan a 1, el valor de R.

imagen

\subsection{Aplicación del modelo de Rescorla y Wagner al experimento de
bloqueo}\label{aplicaciuxf3n-del-modelo-de-rescorla-y-wagner-al-experimento-de-bloqueo}

La figura x muestra el protocolo experimental del procedimiento de
bloqueo. Existen dos grupos, los cuales tienen en común el que se les
presenta un compuesto de un tono y una luz, seguidos por el acceso a
comida. Ambos grupos difieren en que para uno de ellos, al cual
llamaremos el grupo de bloqueo, en una primera fase se le presenta solo
el tono seguido de la comida. El otro grupo, un control, no tiene esta
experiencia. Para el grupo de bloqueo, que recibe en la fase 1 la
experiencia con el tono seguido de la comida, al final de esa fase el
valor (R - Vtono) es casi cero y el valor del tono VT es igual a R. En
lenguaje menos técnico, el tono predice perfectamente la presentación de
la comida.

Para este grupo en la fase en el primer ensayo de esa fase Vtotal = VT +
VL = (R + 0) y consecuentemente (R - Vtotal ) = (R - R +0) = 0 .
Computando la actualización del valor de la luz: VL+1 = VL + a (R -
Vtotal) = 0 + a (1 - 1) = 0

Vemos que no se le asigna valor al elemento luz. En resumen, cuando el
elemento de un compuesto ya predice la presentación del refuerzo, el
otro elemento del compuesto no adquiere valor predictivo, tal y como
puede verse en el resultado de la simulación presentada en la Figura x.

imagen

\subsubsection{Predicción contraintuitiva del modelo de Rescorla y
Wagner}\label{predicciuxf3n-contraintuitiva-del-modelo-de-rescorla-y-wagner}

Cualquier versión del modelo de refuerzo predice que un refuerzo
adicional debe incrementar, aunque sea por un monto muy pequeño, el
valor predictivo de un estímulo. Sin embargo, veamos qué predice el
modelo de Rescorla y Wagner en el siguiente protocolo. A un grupo lo
exponemos a tres fases de entrenamiento. En la primera fase, un tono es
seguido de comida durante 60 ensayos. En una segunda fase, una luz es el
estímulo condicionado y es seguida de comida durante otros 60 ensayos.
En la tercera fase, la final, a los sujetos experimentales se les
presenta el compuesto tono-luz, seguido de comida durante 60 ensayos. Al
inicio de la tercera fase, VL = R; VT = R y Vtotal = 2R, de tal forma
que el error de predicción para ambos estímulos, será R - 2R, por lo que
Vt+1 será un número negativo y veremos un decremento en el valor
predictivo para los dos estímulos. Interesantemente, se ha encontrado
evidencia empírica que respalda esta predicción contraintuitiva del
modelo. La siguiente figura muestra el resultado de la simulación.

imagen

\subsubsection{Aplicación del modelo de Rescorla y Wagner a estudios de
protocolos de
correlaciones}\label{aplicaciuxf3n-del-modelo-de-rescorla-y-wagner-a-estudios-de-protocolos-de-correlaciones}

Un reto importante para la ecuación de Rescorla y Wagner es dar cuenta
de los resultados de los experimentos de Rescorla en los que se manipula
la relación de contingencia: esto es, experimentos en los que se
manipula la probabilidad de la presentación del refuerzo, dada la
presencia o ausencia del EC. Recuerden que en esos experimentos, se
encontró que manteniendo constante la probabilidad de refuerzo en la
presencia del estímulo condicionado, el crédito que se le asigna depende
de la probabilidad de refuerzo en su ausencia. Sin embargo, de acuerdo a
una interpretación literal de la ecuación de Rescorla y Wagner, el error
de predicción para el estímulo condicionado es independiente de la
aparición o no aparición del refuerzo en la ausencia del EC. La solución
propuesta por Rescorla y Wagner para que su modelo de cuenta de estos
hechos empíricos es considerar al contexto en el que se presenta el EC
como un estímulo más. El contexto es el interior del espacio
experimental e incluye, entre otros elementos, la iluminación, el olor y
la textura del espacio. De esta forma, el protocolo de los experimentos
de Rescorla incluye dos estímulos: el compuesto del estímulo
condicionado junto con el contexto; y un segundo estímulo, el contexto
solo. En el caso del procedimiento con igual probabilidad de refuerzo en
la presencia y la ausencia del EC, la ecuación de Rescorla y Wagner
interpreta el experimento como uno de bloqueo en el que el contexto X es
el mejor predictor del refuerzo y termina bloqueando la asignación de
crédito al estímulo condicionado.

Una forma de evaluar su comprensión del modelo de Rescorla y Wagner es
considerar cuál sería su predicción para un experimento con protocolo no
correlacionado (sin correlación entre el EC y R), en el cual los
refuerzos que se presentan durante el intervalo entre ensayos son
señalados con un tercer estímulo diferente al estímulo condicionado.
¿Qué cambios se pueden esperar en la asignación de crédito el estímulo
condicionado?

\subsection{El modelo de Rescorla y Wagner e inhibición
condicionada}\label{el-modelo-de-rescorla-y-wagner-e-inhibiciuxf3n-condicionada}

Hasta este punto, hemos argumentado que de acuerdo al modelo de Rescorla
y Wagner, tanto estímulos como respuestas adquieren un valor que les
permite predecir la presencia de un refuerzo, pero siguiendo este modelo
¿pueden los estímulos/respuestas predecir la ausencia de un refuerzo? En
este apartado le daremos respuesta a esa pregunta.

Poder predecir la \emph{no} ocurrencia de ciertos refuerzos, tiene
importantes ventajas competitivas para el organismo, en particular, le
permite acomodar su distribución de comportamientos de una mejor manera.
La señal de que un depredador no va a aparecer, le permite a la
potencial presa buscar su alimento sin interrupciones; de igual manera,
la señal que predice que no habrá comida, le permite al organismo
reorientar su comportamiento hacia la búsqueda de otros refuerzos. Al
estudio de este fenómeno se le conoce como \emph{inhibición
condicionada}.

El estudio de la inhibición condicionada tardó décadas en despegar por
dos razones. La primera está relacionada con la estructura del modelo
original de aprendizaje por refuerzo, que no permite valores negativos
para el valor de un estímulo. Si la mayor parte del flujo de estímulos y
respuestas no van seguidos de un refuerzo, ¿que se aprende acerca de
estos eventos? Imaginen que se encuentran con una persona paseando a un
perro que los ignora completamente. Consideremos qué predice el modelo
de refuerzo sobre lo que ustedes aprenderán acerca de esa persona. En
este episodio, el perro no era un suceso biológicamente importante -ni
les gruñó, ni les movió la cola- consecuentemente R es igual a cero.
Adicionalmente, la persona era un desconocido que no predice nada, su V
es por lo tanto igual a cero. De acuerdo al modelo de refuerzo, el
cambio en el valor predictivo de la persona (Vx) es una función del
error de predicción (R - Vx), en este caso (0 - 0) y por lo tanto no
habría tampoco ningún cambio en Vx. En otras palabras, si dado un
estímulo, nada se espera y nada se obtiene, ese estímulo no predice
nada.

A diferencia del modelo de refuerzo tradicional, el modelo de Rescorla y
Wagner permite que un estímulo tenga un valor negativo y sea un
predictor de la ausencia de un reforzador. Regresemos a nuestro ejemplo
de una persona paseando a un perro, excepto que esta vez, imaginemos que
el perro les gruñe de forma amenazante. Después de muchos encuentros
similares, la persona paseando al perro se convierte en el predictor de
un perro agresivo. En un siguiente encuentro, la persona que pasea al
perro va acompañada de su pareja y el perro esta vez no les gruñe,
\emph{generando un error de predicción} con valor negativo. Después de
muchos encuentros de este tipo, la pareja de la persona que pasea al
perro se convierte en un inhibidor condicionado, el cual predice la no
ocurrencia del gruñido del perro. Recordando que en la ausencia de un
refuerzo, R es igual a cero, el error de predicción es negativo sólo si
el estímulo neutro aparece en compuesto con un estímulo con valor
positivo. De esa forma V total \textgreater{} 0 y el error de predicción
(R - V total ) \textless{} 0.

En conclusión, de acuerdo a Rescorla y Wagner, un estímulo/ respuesta se
convierte en un inhibidor condicionado, solo si hay un error de
predicción negativo como resultado de presentarlo en compuesto con un
predictor de refuerzo.

La segunda razón que dificulta el estudio de la inhibición condicionada
es la dificultad para distinguir empíricamente entre un estímulo neutro
-es decir, uno que no predice nada- y un estímulo que predice la
ausencia de algo. En este tema, la contribución de Rescorla es también
un punto de partida. En 1969, propuso dos protocolos necesarios para
argumentar y sostener que un estímulo era un inhibidor condicionado.

En un primer protocolo, conocido como de \emph{sumación}, se compara,
por un lado, la respuesta a un estímulo condicionado A con valor
positivo presentado individualmente; con, por otra parte, la respuesta
ante un compuesto del estímulo A acompañado de un estímulo X. Nuestro
objetivo es determinar si X es un inhibidor condicionado. Si la
respuesta al compuesto AX es menor u opuesta a la respuesta observada
ante el estímulo A presentado individualmente, podríamos concluir que el
estímulo X es un inhibidor condicionado. Regresando a nuestro ejemplo,
podemos argumentar que la pareja del paseador de perro es un inhibidor
condicionado, si el perro no gruñe cuando la pareja acompaña al paseador
y sí gruñe cuando este va acompañado únicamente de su paseador. Sin
embargo, Rescorla señala que estos resultados tienen una segunda
interpretación: es posible que la atención dirigida al estímulo X (la
pareja) reduzca la atención dirigida al estímulo A (el paseador)
resultando en la menor respuesta al perro. En resumen, X no sería un
predictor de la ausencia de gruñido (no sería un inhibidor
condicionado), simplemente, X contribuiría a que se ignore a A. Las
siguientes dos figuras muestran el protocolo de sumación y el resultado
de una simulación.

IMAGENES

De acuerdo a Rescorla, para descartar la interpretación alternativa del
protocolo de sumación en términos de atención se requiere de una prueba
adicional. A esta se le conoce como \emph{prueba de retardo} y consiste
en comparar la curva de adquisición de valor predictivo de un estímulo
neutral A presentado individualmente, con la presentación -también
individual- de un estímulo X que se entrenó como un inhibitorio y que
tiene un valor negativo. Si el aprendizaje es más lento para el segundo
estímulo X, podríamos concluir que este se trata de un estímulo
inhibitorio. La figura x muestra los resultados de la simulación con
este protocolo.

IMAGEN

Sin embargo, otra posible explicación de los resultados de la prueba de
retardo hace también referencia a procesos de atención. Es posible que
por el entrenamiento previo, el estímulo X deje de activar los procesos
de atención y que por lo tanto la demora en el aprendizaje de su valor
predictivo se deba a la falta de atención que este recibe en comparación
con la atención que recibe el estímulo novedoso neutral. Sin embargo,
noten que en el protocolo de sumación, la explicación alternativa para
dar cuenta de una menor respuesta del organismo ante el estímulo
compuesto (X+EC) era una mayor atención otorgada al estímulo X, el cual
desviaba la atención del estímulo EC; mientras tanto, en el protocolo de
retardo, la explicación alternativa para dar cuenta de la menor
respuesta ante el estímulo X es una menor atención asignada al estímulo
X debido a la familiaridad con este estímulo. Por lo tanto, Rescorla
propone que para concluir que un estímulo/respuesta es un inhibidor,
este debe pasar tanto la prueba de sumación como la de retardo. No
resulta plausible que un estimulo reciba menos atención en una
circunstancia y el mismo reciba mayor atención en la otra circunstancia:
por lo cual, si el estímulo actúa como inhibidor en ambas
circunstancias, esto significa que su efecto no se debe a meros procesos
atencionales de novedad/habituación, sino que efectivamente, el
organismo considera a este estímulo como un predictor de la ausencia de
un SBI. En otras palabras, dado que la variable de atención tiene
efectos contrarios en las dos pruebas: al encontrar un estímulo que pasa
ambas pruebas, se eliminan las explicaciones alternativas de más y de
menor atención, y se puede considerar a este estímulo como un inhibidor
condicionado.

\subsection{Algunos problemas con el modelo de Rescorla y
Wagner}\label{algunos-problemas-con-el-modelo-de-rescorla-y-wagner}

Dentro de la Psicología, pocos modelos han sido tan exitosos como el de
Rescorla y Wagner en dar cuenta de una amplia gama de resultados, abrir
nuevas rutas de investigación y capturar formalmente explicaciones
alternativas al papel de la contigüidad en la asignación de crédito. Sin
embargo, como ocurre con cualquier otro modelo, hay un número de sus
predicciones que no tienen sustento empírico. Estas fallas han dado
lugar a extensiones de modelos y a modelos alternativos que veremos en
otro capítulo. A continuación presentamos dos de las predicciones
erróneas. Seleccionamos estas por su fácil comprensión y por ser las que
han dado lugar a modelos alternativos.

\subsubsection{Supuesto de que la extinción reduce el valor de un
estímulo/respuesta a
cero}\label{supuesto-de-que-la-extinciuxf3n-reduce-el-valor-de-un-estuxedmulorespuesta-a-cero}

En extinción, el refuerzo que previamente seguía a un estímulo o
respuesta deja de presentarse. En este caso R cambia de un valor de 1 a
0. El modelo asume que no habrá un error de predicción cuando el valor
del estímulo sea igual a R, esto es, cuando el valor del estímulo sea
también cero. Consecuentemente, para Rescorla y Wagner, el impacto de un
estímulo que fue extinguido, debe ser el mismo que el de un estímulo
neutral dado que para ambos estímulos V = 0.

Sin embargo, hay una multitud de reportes que señalan que el mero paso
del tiempo produce una recuperación espontánea del efecto que
originalmente tenía un estímulo que ha atravesado un proceso de
extinción. Adicionalmente, se ha encontrado que estímulos previamente
extinguidos adquieren valor predictivo más rápido que estímulos
neutrales, a pesar de que se supone que ambos inician con un valor igual
a cero. Esta literatura y los modelos para dar cuenta de ella la
presentamos en el capítulo sobre extinción.

\subsection{Inhibición latente}\label{inhibiciuxf3n-latente}

Consideremos el siguiente protocolo experimental. En una primera fase, a
un grupo se le presenta, durante 60 ensayos, un estímulo neutral que no
es seguido por un refuerzo; y en una segunda fase, la presentación de
este mismo estímulo sí va seguida de un refuerzo. A un segundo grupo,
solo se les presenta la segunda fase, sin darle la experiencia con el
estímulo solo. A este protocolo se le conoce como inhibición latente. De
acuerdo a Rescorla y Wagner, el valor de los dos estímulos debería ser
el mismo. Sin embargo, una enorme literatura reporta que la
preexposición a un estímulo sin refuerzo demora la subsecuente
adquisición de valor cuando ese estímulo va acompañado de un refuerzo.

El fenómeno de la inhibición latente ha generado una importante
alternativa teórica al modelo de Rescorla y Wagner, la cual revisaremos
en otro capítulo y que pone el énfasis en la \emph{atención} asignada a
un estímulo cuando este es seguido de un refuerzo inesperado. Inhibición
latente sería el resultado de la falta de atención que se le asigna a un
estímulo que fue presentado por muchos ensayos sin ser seguido por
ningún refuerzo, y que por lo tanto, desde la perspectiva del organismo,
no predice que algo importante ocurrirá.

\bookmarksetup{startatroot}

\chapter{Acción Como Elección}\label{acciuxf3n-como-elecciuxf3n}

Todas las variantes de la ley del efecto nos dicen que las respuestas
que son seguidas por un refuerzo incrementan en frecuencia. Si esta
fuera la única información que nos proporciona este concepto,
difícilmente le asignaríamos el estatus de una una ley y su utilidad
para entender el comportamiento sería muy limitada. En un escenario
aplicado, para la madre que quiere modificar la conducta de un hijo, la
sola recomendación de reforzar la conducta que desea incrementar no
resulta totalmente satisfactoria. La madre quisiera saber si es
necesario reforzar cada instancia de la respuesta o solo algunas de
ellas, quisiera saber si el refuerzo 20 tiene el mismo impacto que el
refuerzo 10 y, finalmente, quisiera saber si el seguimiento de distintas
reglas para la entrega del refuerzo marca una diferencia.

En las notas sobre programas de refuerzo, le dimos ya una respuesta a la
última pregunta: presentamos el comportamiento característico asociado
con distintas reglas de entrega de refuerzo y, en particular, resaltamos
que en programas de razón variable los organismos responden a una tasa
más alta que en los programas de intervalo variable. En el escenario
aplicado, la madre preguntaría si para conseguir que su hijo emita la
respuesta deseada a cierta tasa, el valor de la razón debe ser de 5, 10,
50, 200 o 500 respuestas por refuerzo. La madre también quisiera saber
qué diferencia hace para el comportamiento observado que la oportunidad
de obtener un refuerzo sea de 1 en cada 5, 10, 20, 100, 200 o 500
minutos.

Por lo tanto, lo que le hace falta a la ley del efecto con la que
iniciamos esta nota es especificar la función que describe la relación
entre la tasa de respuesta de un organismo y una medida del refuerzo.
Para propósitos de estas notas, la medida del refuerzo será la tasa de
ocurrencia del reforzador:

\[R_i = f (r_i)\]

Un primer paso para especificar la función aludida es encontrar la
relación empírica entre la tasa de respuesta de los organismos y
distintos valores dentro de programas de intervalo variable y de razón
variable.

\section{Funciones de Respuesta para programas de intervalo
variable}\label{funciones-de-respuesta-para-programas-de-intervalo-variable}

Catania y Reynolds (1968) publicaron el primer estudio sistemático en el
que se estudió la relación, ``en equilibrio'', entre tasas de respuesta
y tasas de refuerzo que resultan de variar los valores del programa de
intervalo. Al hablar de ``equilibrio'', se hace referencia a un
protocolo en el que se entrena al animal con un valor de un programa IV
y se ejecutan el número de sesiones necesarias para alcanzar un
comportamiento estable; finalmente, los datos que se analizan son los
correspondientes a los últimos días de exposición. El procedimiento se
repite para cada uno de los programas IV estudiados. De esta manera, se
determina la relación entre la tasa de refuerzo de distintos intervalos
(valor de entrada de la función) y la tasa de respuesta de los
organismos (valor de salida de la función).

imagen

En la siguiente figura se muestran los resultados del estudio de Catania
y Reynolds. Puede verse que hay una relación de ganancias decrecientes
entre la tasa de respuesta y la tasa de refuerzo que produce. La función
crece rápidamente conforme incrementa la tasa de refuerzo (es decir,
conforme aumenta el número de refuerzos por unidad de tiempo), hasta
alcanzar un punto después del cual, incrementos subsecuentes en la tasa
de refuerzo tienen un efecto cada vez menor sobre la tasa de respuesta
del organismo. Esta función se encontró posteriormente en docenas de
estudios (de Villiers y Herrnstein).

\subsection{Relación Entre Tasas Absolutas y Relativas de
Respuesta}\label{relaciuxf3n-entre-tasas-absolutas-y-relativas-de-respuesta}

Para Herrnstein (1970) fue claro que cualquier función que relacione la
tasa de respuesta (como valor de salida) a la tasa de refuerzo (como
valor de entrada) debe satisfacer dos condiciones: * primero, dar cuenta
de la relación obtenida en los programas de intervalo variable cuando
hay una sola opción de respuesta disponible para el organismo y *
segundo, ser consistente con la ley de igualación, cuando hay dos o más
opciones de respuesta para el organismo.

Recordemos que para los modelos de refuerzo, los animales no llevan un
registro de la frecuencia relativa con la que deben emitir cada
respuesta, en lugar de ello, estos modelos postulan que para el
organismo cada respuesta adquiere un valor por separado, los cuales
subyacen a los patrones de respuesta observados. El valor de cada
respuesta varía en función del refuerzo que ésta produce. Por otro lado,
la tasa relativa con la que el organismo emite cada respuesta es el
resultado de una comparación entre el valor adquirido por cada una de
las distintas respuestas. El objetivo, por lo tanto, es encontrar la
función que describe la relación entre los tres valores: el valor
adquirido por cada respuesta; la tasa con la que el organismo emite cada
respuesta; y las tasas de refuerzo que resultan de cada una de las tasas
de respuesta:

\[R_i = f (r_i)\]

Y usando la función \emph{f}, derivar la relación entre las tasas
relativas de respuesta y las tasas relativas de refuerzo. La función
\emph{f} la vamos a evaluar por su éxito para dar cuenta de las tasas
relativas de respuesta (siguiendo el patrón de la ley de igualación)
observadas en programas concurrentes IV - IV. Para ello, debemos tener
presente que la programación de los refuerzos en lo programas
concurrentes puede hacerse de dos formas:

\begin{itemize}
\tightlist
\item
  La primera es manteniendo constante la tasa de refuerzo a lo largo del
  experimento, pero variando la proporción asignada a cada respuesta en
  diferentes condiciones experimentales. Por ejemplo, se pueden
  programar 60 refuerzos por hora, al mismo tiempo que se establecen
  diferentes condiciones experimentales en las que se generan las
  siguientes distribuciones de refuerzos para las dos respuestas
  posibles: 50-10, 40-20, 30-30, 20-40 y 10-50.
\item
  La segunda es programando una tasa de refuerzo fija para una de las
  respuestas a lo largo del experimento, a la vez que se varía la tasa
  de refuerzo para la segunda respuesta en diferentes condiciones
  experimentales.
\end{itemize}

La siguiente figura muestra los resultados de un experimento que compara
los resultados de las dos formas de estructurar los programas
concurrentes:

imagen

La figura del panel izquierdo muestra la tasa de respuesta de las dos
opciones bajo la condición donde la tasa de refuerzo total es constante.
La figura del panel derecho muestra la tasa de respuesta de las dos
opciones bajo la condición en la cual la tasa de refuerzo para una
respuesta (símbolos negros) es constante, mientras que la tasa de
refuerzo para la otra respuesta es variable (círculos abiertos). Podemos
ver que en la primera condición, la tasa de respuesta está linealmente
relacionada a su tasa de refuerzo. A mayor tasa de refuerzo, mayor tasa
de respuesta. Mientras tanto, en la segunda condición, la tasa de
respuesta a la opción con refuerzo fijo disminuye en función del aumento
en el refuerzo obtenido en la otra opción de respuesta.

En otras palabras, la diferencia sútil pero crucial entre las dos
condiciones reside en cómo la tasa de respuesta de una opción varía con
relación a la tasa de respuesta de la otra opción dentro de una misma
condición.

En la primera condición (refuerzo total constante), si la tasa de
refuerzo de una opción aumenta, la tasa de refuerzo de la otra opción
decrementa proporcionalmente (dado que el total se encuentra fijo). Esto
genera una compensación perfectamente lineal a nivel de las tasas de
refuerzo: y las tasas de respuesta reflejan esta relación lineal porque
esencialmente ambas opciones compiten por un pool de respuestas fijo.

En la segunda condición (una opción con refuerzo fijo, la otra con
refuerzo variable), el aumento a la tasa de refuerzo para la opción
variable disminuye la tasa de respuesta para la opción fija. A pesar de
que la tasa de refuerzo de la opción fija no ha cambiado en absoluto,
las respuestas a ella disminuyen de todas formas cuando la segunda
opción se vuelve más recompensante. Esto demuestra que el organismo no
mantiene meramente un registro de las tasas de refuerzo absolutas, sino
que ejecuta una computación más compleja sobre el valor relativo de cada
opción dentro del contexto más amplio.

\subsection{Posibles Funciones de
Refuerzo}\label{posibles-funciones-de-refuerzo}

La primera función que podemos considerar tiene su origen en una
propuesta de Skinner. De acuerdo a esta, la tasa de respuesta es
directamente proporcional a la tasa de refuerzo.

\[R_i = kr_i\]

La ecuación describe una línea recta, donde el parámetro \textbf{k} es
su pendiente y representa la traducción de un refuerzo en un incremento
fijo en la tasa de respuesta. (Puede trabajar con el correspondiente
simulador).

Para determinar si esta primera función es consistente con la ley de
igualación, insertamos la función en la computación de tasas relativas
de respuesta. Asumiendo que el refuerzo es el mismo para las dos
opciones, el parámetro \textbf{k} se cancela y obtenemos la ecuación de
igualación.

\[\frac {R_1} {(R_1 + R_2)} = \frac {kr_1} {(kr_1 + kr_2)}\]

La función que postula proporcionalidad entre respuestas y refuerzos es
consistente con los resultados presentados en la Fig x. Sin embargo, la
función \[R_i = kr_i\] propone que la tasa de una respuesta depende
únicamente del refuerzo que ésta produce. Por lo tanto, la tasa de una
respuesta no debería de cambiar cuando su refuerzo es constante y tan
solo se manipula otra fuente de refuerzo. Sin embargo, en el panel
derecho de la figura x observamos que la tasa de respuesta con el
refuerzo constante decrece conforme incrementa el refuerzo para la
respuesta alternativa: fenómeno que se conoce como \textbf{contraste
conductual}.

La siguiente es una segunda función de respuesta, consistente con el muy
replicado resultado de contraste conductual:

\[R_1 = \frac {kr_1} {(r_1 + r_2)}\]

La función nos dice que la tasa de una respuesta es una función de la
tasa relativa de refuerzo que recibe esta respuesta. Como puede verse en
el simulador, si la tasa de refuerzo \(r_2\) para la otra opción es
constante, la forma de la función de la tasa de respuesta \(R_1\) es de
ganancias decrecientes, igual a los datos empíricos reportados en la
figura x. Por otra parte, si \(r_1\) es constante, conforme incrementa
el valor de \(r_2\), la tasa de de respuesta \(R_1\) decrementa,
reproduciendo el patrón de contraste conductual.

Esta función es consistente con el resultado de la ley de igualación.
Sustituyendo y cancelando el parámetro \emph{k} y los denominadores,
\begin{equation}
\frac{R_1}{R_1 + P_2} = \frac{\frac{kr_1}{r_1 + r_2}}{\frac{kr_1}{r_1 + r_2} + \frac{kr_2}{r_1 + r_2}} = \frac{kr_1}{kr_1 + kr_2} = \frac{r_1}{r_1 + r_2}
\end{equation}

Esta segunda función da cuenta de los resultados obtenidos en programas
concurrentes. Sin embargo, veamos qué predice la misma función para
experimentos en los que solo se refuerza una de las respuestas, como es
el caso de los datos del experimento de Catania y Reynolds. En la
ecuación:

\[R_1 = \frac {kr_1} {(r_1 + r_2)}\]

el valor de \(r_2\) es cero, pues solo hay una respuesta (r\_1) y la
ecuación se reduce a

\[R_1 = \frac {kr_1} {r_1 }\]

Por lo tanto, terminamos con la función:

\[R_1 = k\]

la cual nos dice que la tasa de respuesta es una constante y que por
ende, la respuesta es insensible a la tasa de refuerzo. Sin embargo,
esta conclusión matemática no corresponde a los resultados reportados en
la revisión de Villiers y Herrnstein sobre múltiples experimentos
similares a los de Catania y Reynolds. En ella, se observa que para
protocolos con una única opción de respuesta, dos cosas son ciertas: la
tasa de respuesta NO crece linealmente en función de la tasa de
refuerzo, contrario a lo que sugiere la primera función de Skinner; y,
de igual manera, la tasa de respuesta TAMPOCO es una constante
insensible a variaciones en la tasa de refuerzo, contrario a lo que
sugiere la segunda función de refuerzo.

\subsection{La ley del Efecto
Relativa}\label{la-ley-del-efecto-relativa}

En un artículo publicado en 1970, Herrnstein propuso una versión de la
ley del efecto que simultáneamente daba cuenta de dos de las
regularidades empíricas más robustas: la igualación en programas
concurrentes y la función de ganancias decrecientes entre tasa de
respuesta y la tasa de refuerzo. Propuso un planteamiento que en su
momento fue una gran salto:

\emph{Todo comportamiento es una instancia de una elección y el papel
del refuerzo es redistribuir el comportamiento.}

La propuesta descansa en tres supuestos:

\begin{enumerate}
\def\labelenumi{\arabic{enumi}.}
\tightlist
\item
  Los organismo están en constante actuar y no existen vacíos
  conductuales. En toda situación experimental hay al menos dos
  respuestas, una, \(R_1\), que medimos directamente (como picar una
  tecla, por ejemplo) y un conjunto de respuestas que no medimos
  directamente (como dar vueltas, aletear, picar el piso, etcétera) y
  que que Herrnstein llamó \(R_o\). La suma de estas dos respuestas
  componen la totalidad del comportamiento \emph{k}.
\end{enumerate}

\[R_1 +Ro = k\]

\begin{enumerate}
\def\labelenumi{\arabic{enumi}.}
\setcounter{enumi}{1}
\item
  En adición al refuerzo programado en una situación experimental,
  siempre hay otros refuerzos disponibles para el organismo, llamados
  \emph{ro}.
\item
  Los organismos igualan la frecuencia relativa de la respuesta
  registrada y de todas las otras respuestas con la tasa relativa de
  refuerzo de todas las respuestas en una situación experimental.
\end{enumerate}

\[\frac{R_1}{R_1 + R_0}= \frac {r_1} {(r_1 + r_o)}\]

\[R_1+R_0 = k\]

\[\frac{R_1}{k}= \frac {r_1} {(r_1 + r_o)}\]

\[R_1= \frac{kr_1} {(r_1 + r_o)}\]

Recordemos que \(k\) es un parámetro que se estima estadísticamente y
representa la suma de todos los comportamientos medida en unidades
\(R_1\): frecuentemente respuestas por minuto. El valor de \(k\) depende
de la topografía (la forma específica) de la respuesta \(R_1\) que se
mide, así como del tiempo que toma ejecutarla. Picar una tecla es mucho
más fácil para una paloma de lo que es apretar una palanca para una
rata: como consecuencia, el valor de \(k\) es mayor cuando se mide el
comportamiento de picar una tecla que cuando se mide el comportamiento
de apretar una palanca.

\(r_o\) es un segundo parámetro que se estima estadísticamente y que
representa la tasa de otros refuerzos no controlados en el experimento.
\(r_o\) se mide en las mismas unidades que \(r_1\): esto es, el número
de refuerzos recibidos por unidad de tiempo.

En el simulador, ustedes pueden ver cómo varía la forma de la función
que relaciona la tasa de respuesta a la tasa de refuerzo cuando se varía
el valor de \(r_o\) dentro de programas de intervalo variable. En
general, estos modelos exponen cómo el número de refuerzos por unidad de
tiempo generados por una respuesta (dentro de un programa de intervalo
variable X) influye sobre la tasa con la que el organismo emite esa
respuesta; al mismo tiempo, la tasa de refuerzo que generan las
respuestas no controladas (R\_o) influye sobre la tasa relativa de la
respuesta evaluada (R\_1).

\begin{itemize}
\item
  La función \(R_1\) es de \emph{ganancias decrecientes}, esto es, el
  impacto de un refuerzo adicional es \textbf{mayor} cuando \(r_1\) es
  pequeño y va \textbf{decreciendo} conforme \(r_1\) incrementa. Ver
  figura.
\item
  La función nos muestra que el impacto de un refuerzo contingente sobre
  una respuesta \emph{depende del contexto de refuerzo}. A medida que
  aumenta \(r_o\), el impacto de incrementar \(r_1\) es menor. Con
  valores de \(r_o\) muy pequeños, un contexto de refuerzo muy pobre, la
  tasa de respuesta es muy sensible a cambios en la tasa de refuerzo
  \(r_1\) y justo lo opuesto ocurre con un contexto de refuerzo muy
  rico, donde \(r_o\) es muy grande. Para una persona en pobreza
  extrema, pequeños cambios en el monto de su pensión tienen gran
  impacto en su comportamiento, pero no así para una persona en un
  entorno de riqueza. Ver figura.
\end{itemize}

\subsection{Impacto sobre la modificación de la
conducta}\label{impacto-sobre-la-modificaciuxf3n-de-la-conducta}

La ley del efecto relativo ha tenido un gran impacto sobre la práctica
de la modificación de la conducta. Considere un caso en el que se
pretende reducir un comportamiento indeseable en un niño, como podría
ser un comportamiento agresivo. De acuerdo a la versión original de la
ley del efecto, una posibilidad es eliminar el refuerzo para ese
comportamiento. Desafortunadamente, el refuerzo para la conducta
agresiva del agresor puede ser la reacción de sumisión y respeto que
este comportamiento induce en el niño agredido, sobre la cual no es
fácil tener un control. Otra posibilidad es castigar al niño agresivo,
arriesgando el surgimiento de otras respuestas indeseables. El modelo de
Herrnstein proporciona una alternativa viable y exitosa: esta consiste
en seleccionar y reforzar un comportamiento incompatible con el
agresivo. Por ejemplo, se le pide a la maestra reforzar y jugar un juego
de mesa con el niño. Para involucrarse plenamente en el juego de mesa,
el niño tiene que sacrificar otros de sus comportamientos, entre ellos,
sus comportamientos agresivos.

La estrategia puede extenderse a problemas tan severos como el consumo
excesivo de bebidas alcohólicas. Una de las consecuencias del
alcoholismo es el creciente aislamiento de la persona, con la asociada
reducción en el contexto de refuerzo social. Esto genera un círculo
vicioso, puesto que la reducción en el contexto de refuerzo social
aumenta el impacto del refuerzo asociado con la bebida alcohólica. De
acuerdo al modelo de Herrnstein, una estrategia exitosa sería recuperar
la vida social de la persona, incrementando así los refuerzos que
conforman su contexto cotidiano.

Similarmente, si se desea incrementar un comportamiento, la estrategia
sugerida por el modelo de Herrnstein consiste en reducir las otras
fuentes de refuerzo disponibles para la persona. El impacto de reforzar
la conducta de estudio de un niño es mayor si se eliminan las opciones
de la televisión y de un celular.

En resumen, la ley del efecto relativa, al dirigir el estudio del
comportamiento al estudio de la elección, proporciona una novedosa
alternativa a las prácticas tradicionales de la modificación de la
conducta. En lugar de simplemente eliminar o castigar conductas
indeseables, se busca promover alternativas deseables y modificar el
contexto de refuerzo para influir en las elecciones del individuo.

\subsection{Evaluación}\label{evaluaciuxf3n}

Empíricamente, la ley del efecto relativo de Herrnstein es enormemente
exitosa en su descripción de la relación entre tasa de respuesta y la
tasa de refuerzo dentro de programas de intervalo variable (de Villiers
y Herrnstein). Sin embargo, la ecuación es algo más que un ejercicio
para el ajuste de datos empíricos, esta es el resultado de un conjunto
de supuestos teóricos que se reflejan en la interpretación de los dos
parámetros de la ecuación \(k\) y \(r_o\).

Las siguientes son algunas de las predicciones teóricas del modelo:

\begin{itemize}
\tightlist
\item
  La suma del total de respuestas, capturada por \(k\), debe ser
  constante e independiente de variables que afecten el valor del
  refuerzo contingente y el valor de \(r_o\), tales como manipulaciones
  motivacionales y la calidad del refuerzo contingente. En otras
  palabras, cuando se dan cambios en el valor de los refuerzos, lo que
  puede cambiar es la frecuencia relativa de las distintas respuestas,
  pero el número total de respuestas potencialmente realizables por el
  organismo debe permanecer constante.
\item
  Las manipulaciones en el tipo y la magnitud del refuerzo contingente
  deben ser capturadas por cambios en \(r_o\).
\item
  Agregar otra fuente de refuerzo debe reducir el valor del parámetro
  \(r_o\). \#\#\# Quedamos de elaborar más/ejemplificar estos últimos
  puntos..
\end{itemize}

La evidencia acerca de los supuestos teóricos de la ley del efecto
relativo no son consistentemente favorables (Dallery y Soto, 2004), en
particular, en varios experimentos se reportan cambios en el parámetro
\(k\) cuando se dan cambios en la magnitud de refuerzo: es decir que no
sólamente se redistribuye el número de respuestas asignado a las
distintas opciones de comportamiento, sino que en efecto, el número
total de respuestas que el organismo emite dentro de una cierta duración
temporal incrementa. Estos resultados, acompañados del hecho de que la
ecuación no se ha aplicado a los resultados obtenidos en programas de
razón variable, han llevado a la derivación de la ecuación de ganancias
decrecientes a partir de otros supuestos, tema que abordaremos en las
siguientes notas. Podemos concluir que, sin duda, la ley del efecto
relativo de Herrnstein no solo generó una gran cantidad de evidencia
empírica, sino que adicionalmente brindó las bases para la gran mayoría
de los modelos de acción. Estas bases serán desarrolladas a detalle
posteriormente, pero pueden resumirse en la siguiente lista:

\begin{enumerate}
\def\labelenumi{\arabic{enumi}.}
\item
  El estudio de la acción es el estudio de la elección.
\item
  La acción observada se mide por su tasa de ocurrencia y refleja la
  distribución total del comportamiento posible.
\item
  El efecto del refuerzo es cambiar la distribución del comportamiento.
\item
  El efecto del refuerzo depende del contexto de otros refuerzos
  presentes, incluyendo los refuerzos no medidos directamente.
\item
  La regla que gobierna la distribución del comportamiento es la
  igualación de la frecuencia relativa de la respuesta a la tasa
  relativa de refuerzo con la que ésta se encuentra asociada.
\end{enumerate}

\bookmarksetup{startatroot}

\chapter{Elección Recurrente:
Igualación}\label{elecciuxf3n-recurrente-igualaciuxf3n}

Consideren las siguientes situaciones: a un agente se le presentan dos
bolsas, y se le \textbf{informa} que una contiene \$100,000 y la otra
\$100 a la vez que se le pide que opte por una de ellas; en un segundo
escenario, al agente se le lleva a un restaurante que no volverá a
visitar y se le pide elegir entre uno u otro platillo. En ambos
ejemplos, la elección puede hacerse exclusivamente a partir del valor de
las opciones en el momento de la elección, y anteriormente hemos visto
que el agente selecciona la opción con mayor valor en el momento de
decisión. Estas situaciones de elección son instancias de sistemas
abiertos sin retroalimentación y siguiendo a Gallistel (fecha), les
llamamos protocolos de \emph{optar}. En las últimas décadas, la mayoría
de la investigación psicológica con participantes humanos ha utilizado
este tipo de protocolo.

Consideren ahora un experimento similar: en lugar de informar al agente
acerca del contenido de las bolsas en un inicio, a este se le presentan
las opciones como bolsas cerradas, y a través de múltiples iteraciones,
el agente tiene que explorar y \textbf{aprender} acerca de los montos de
dinero de las distintas bolsas o de la calidad de las distintas opciones
ofrecidas por el restaurante. La presentación recurrente de las
oportunidades de elección crea nuevos problemas de adaptación para el
agente. Como vimos en las notas anteriores, el agente debe resolver el
dilema de exploración - explotación y debe determinar si las
consecuencias de las opciones cambian como una función de las elecciones
y el paso del tiempo, o si estas son fijas e independientes de las
elecciones y del tiempo. En un ejemplo no ya de una elección entre
varios platillos de un restaurante sino de una elección entre acudir a
uno de varios restaurantes, el agente debe visitar los distintos locales
un buen número de veces antes de tomar la decisión de pasar a explotar
uno. Sin embargo, aún tras haber estimado la opción con mayor valor y
haber comenzado a explotar una opción elegida, el agente debe mantener
cierta flexibilidad para modificar su elección, en aras de poder
determinar si la calidad de los restaurantes varía aleatoriamente con el
paso del tiempo (debido a factores como los cambios de cocinero, por
ejemplo). En estos casos, la regla de elegir la opción que
instantáneamente tiene más valor no es la que a largo plazo proporciona
la mayor cantidad de refuerzos. En los experimentos con estos protocolos
observamos una regla de respuesta probabilística.

\subsection{Elección Recurrente}\label{elecciuxf3n-recurrente}

Fuera del laboratorio, lo común para los humanos y para muchas otras
especies son situaciones en las cuales los individuos pueden elegir
entre dos o más acciones o parcelas de forma repetida y continua, y en
las cuales las elecciones alteran las opciones futuras de refuerzo.
Estos problemas de adaptación son ejemplos de sistemas de
retroalimentación cerrados y, siguiendo a Gallistel, les llamamos
\emph{problemas de asignación} de respuestas, tiempo o esfuerzo. Un
estudiante a lo largo del día va asignando su tiempo a diferentes
actividades: desayunar, transportarse, pasar tiempo en el salón de
clases, estudiar en la biblioteca, conversar con amistades, ver a su
pareja, ejercitarse. Al final del día, habrá una distribución de tiempos
asignados a las diferentes actividades. Resulta importante considerar
que cada una de estas actividades es en sí otro espacio de posibles
acciones a las que se les puede dedicar tiempo. Por ejemplo, cuando
están en el salón de clases, pueden atender lo que presenta el profesor
o pueden ver las noticias en su celular, mandar un WA o fantasear. De
esta forma, podemos estudiar problemas de asignación a diferentes
escalas temporales, pero todas bajo el mismo esquema.

Un ejemplo adicional lo proporciona el forrajeo de una abeja que
enfrenta dos diferentes parcelas con flores con polen. En este entorno,
mientras más tiempo pasa la abeja visitando las flores de una de las
parcelas, la disponibilidad del polen dentro de la misma va
disminuyendo; al mismo tiempo, las flores en la otra parcela siguen
llenas de polen. La abeja enfrenta dos problemas de adaptación: el
primero es decidir cuánto tiempo agregado dedicarle a cada una de las
dos parcelas como una función de la distribución de flores con alimento
en cada una de las parcelas. Normalmente, en el contexto de asignación,
la regla es distribuir el comportamiento a lo largo del tiempo de una
forma que produzca la mayor ganancia posible. El segundo problema de
adaptación es la decisión de cuándo salirse de una de las parcelas para
visitar la otra. En estas notas, nos centraremos en el estudio del
primer problema de adaptación: la distribución de respuestas y tiempos.

La forma más sencilla de estudiar experimentalmente protocolos de
elección recurrente se desarrolló en el laboratorio de Skinner y se
conoce como \emph{programas de refuerzo concurrentes}. El protocolo
consiste en presentarle a un organismo dos o más opciones de respuesta
-teclas iluminadas en el caso de las palomas- que se encuentran
disponibles todo el tiempo y que siguen programas individuales e
independientes de refuerzo. (Figura). En estas notas, revisaremos los
resultados obtenidos en el estado de equilibrio, una vez que los agentes
han aprendido acerca de las consecuencias de cada opción de respuesta.
La variable que se estudia es la distribución de respuestas o tiempos
asignados.

\[
\frac{R_1} {(R_1 +R_2)}
\]

imagen

En los estudios que reportamos, el animal es expuesto a un par de
programas de refuerzo dentro de sesiones diarias hasta que la
distribución de respuestas a las distintas opciones disponibles se
vuelve estable y deja de cambiar día con día. Esto toma entre 30 y 45
días. Esta rutina se repite para todos los pares de programas que se
están estudiando. De cada par, se usan para el análisis los últimos
cinco días en los que las elecciones son estables.

\subsection{La Ley de Igualación}\label{la-ley-de-igualaciuxf3n}

En 1961, Richard Herrnstein (ver foto)

imagen

reportó los resultados del primer estudio con programas concurrentes, en
el que la respuesta de picar una de dos teclas era reforzada de acuerdo
a un programa de intervalo variable. Recuerden que en estos programas el
refuerzo se presenta tras la primera respuesta después de que haya
transcurrido un tiempo aleatorio desde el último refuerzo. Un detalle
muy importante que debe tenerse presente bajo este tipo de programa es
que una vez que un refuerzo está disponible, la oportunidad de obtenerlo
se retiene hasta que el animal responde a esa opción.

Herrnstein encontró que en éstos programas la tasa relativa de
respuestas (su proporción) iguala la tasa relativa de reforzadores
obtenidos:

\[
\frac {R_1} {(R_1 + R_2)} = \frac {r_1} {(r_1 + r_2)}
\]

El resultado es muy robusto y se ha reportado en un sinnúmero de
especies. A esta relación entre tasas relativas de respuesta y refuerzo
se le conoce como la \emph{ley de igualación} y en la última década del
siglo pasado fue la ley más citada en la literatura psicológica. En la
siguiente figura pueden verse los resultados de tres palomas: cada punto
representa los datos de los últimos cinco días para cada par de valores
de los programas de intervalo variable. La proporción de respuestas
(tasas relativas) va de cero a uno. (Figura).

imagen

Igualación sería un resultado trivial, si por cada refuerzo hubiese solo
una respuesta, sin embargo, el patrón de igualación en los refuerzos
también se puede obtener con un rango muy amplio de tasas relativas de
respuesta que van más allá de la tasa específica del patrón de
igualación.

Una forma más directa de estudiar la relación entre patrones de
respuesta y patrones de refuerzo en protocolos de elección recurrente,
es estudiando el \textbf{tiempo} asignado por los organismos a las
diferentes opciones de refuerzo, en lugar de estudiar el número de
respuestas discretas asignadas a las distintas opciones. Desde este
enfoque, Rachlin y Baum estudiaron el comportamiento de las palomas:
para ello, emplearon un espacio rectangular con un piso conectado a
interruptores que permite medir el tiempo que una paloma pasa en cada
lado del espacio rectangular. En cada uno de los dos extremos del
espacio experimental había un comedero que asignaba comida de acuerdo a
programas independientes de refuerzo de intervalo variable. En este
experimento también se encontró que el tiempo relativo asignado por el
organismo a un lugar iguala el refuerzo relativo obtenido en dicho
lugar.

\[
\frac {T_1} {(T_1 + T_2)} = \frac {r_1} {(r_1 + r_2)}
\]

\section{Desviaciones de
Igualación}\label{desviaciones-de-igualaciuxf3n}

La igualación de las tasas relativas de respuesta al valor de las tasas
relativas de refuerzo es un fenómeno muy robusto cuando ambas opciones
de respuesta son reforzadas de acuerdo a programas de intervalo
variable, sin embargo, se han encontrado desviaciones respecto al patrón
de igualación cuando uno de los programas de refuerzo se cambio a otra
regla, o cuando se establecen distintos tipos de reforzadores para las
dos respuestas. Baum () reconoció dos tipos de desviaciones de
igualación: Introducir el ejemplo de ``ver Salir a alguien con una
cantidad de frijoles de diferente tipo''. ¿Es el resultado de
preferencias o de precios?

\begin{enumerate}
\def\labelenumi{\arabic{enumi}.}
\tightlist
\item
  \emph{Sesgos}. Si en una visita al supermercado les ofrecen probar,
  sin ningún costo, frijoles negros o bayos, algunos de Uds. preferirán
  la prueba de los frijoles negros. Dada esa preferencia, si compran
  frijoles y ambos tienen el mismo precio, comprarán los frijoles
  negros. Sin embargo, qué frijoles deciden comprar depende de la
  diferencia en su precio modelada por su preferencia. Cuando los
  reforzadores para las dos respuestas son diferentes, por ejemplo,
  cuando una de las dos variedades de frijoles negros o bayos les brinda
  mayor satisfacción debido a su sabor particular, es posible que exista
  una preferencia por uno de ellos: esta preferencia tendrá un impacto
  sobre cada combinación de razones de refuerzo. Cuando esta razón es
  igual pero todavía se presentan diferencias en la tasa relativa de
  refuerzo, esta diferencia es un indicador del sesgo del organismo en
  favor de uno de los reforzadores. Los resultados se verían como los de
  la figura x: en ella, la tasa relativa de respuesta se aleja de 0.5,
  aún cuando la tasa relativa de refuerzo es igual para las dos
  opciones.
\end{enumerate}

imagen

\begin{enumerate}
\def\labelenumi{\arabic{enumi}.}
\setcounter{enumi}{1}
\tightlist
\item
  \emph{Sensibilidad}. En una misma visita al supermercado, un mismo
  producto que desean comprar es ofrecido por dos marcas distintas a
  precios diferentes: uno cuesta \$11.00 y el otro \$5.50. La diferencia
  en precio es de 2 a 1, sin embargo, los valores numéricos son
  difíciles de discriminar y para algunos de Uds. esta diferencia será
  percibida como de 3 a 1, mientras que para otros, la diferencia se
  percibirá como de 1.5 a 1. Sensibilidad es una segunda desviación de
  igualación, que ocurre cuando los organismos no son linealmente
  sensibles a la diferencia entre las tasas de refuerzo. Esto puede
  deberse a distinciones en la importancia de las diferencias en el
  valor de las opciones o a la dificultad para discriminar entre ellas.
  La figura x muestra como se vería la relación entre tasas relativas de
  respuesta y de refuerzo bajo distintos valores de sensibilidad. Cuando
  la tasa relativa de respuesta no es muy sensible a las tasas de
  refuerzo para cada respuesta, observamos valores cercanos a la
  indiferencia (panel de la izquierda en la figura) y a este resultado
  se le conoce como sub igualación. Cuando la tasa relativa de
  respuestas sobrevalora las diferencias en las tasas de respuesta,
  observamos que se prefiere mayoritariamente la mejor opción (panel
  derecho en la figura) y a este fenómeno se le conoce como
  sobre-igualación.
\end{enumerate}

En resumen, el sesgo hace referencia a la preferencia del organismo por
una de las opciones, la cual tiene un efecto multiplicativo al de la
tasa de ocurrencia de los reforzadores para determinar las tasas de
respuesta. El otro factor de desviación respecto a igualación es la
sensibilidad del agente ante las diferencias en las tasas de ocurrencia
de los refuerzos de las distintas opciones. Vamos a asumir que el sesgo
y la sensibilidad varían independientemente el uno del otro.

imagen

\subsection{Ley generalizada de
Igualación}\label{ley-generalizada-de-igualaciuxf3n}

Para modelar el sesgo y la sensibilidad, Baum propuso una extensión de
la ley de igualación que se conoce como la \emph{Ley Generalizada de
Igualación} y que captura las dos clases de desviaciones revisadas en la
sección anterior. Primero propuso expresar la ley en términos de razones
entre las distintas respuestas y los distintos refuerzos; y no de
proporción entre una respuesta y el total de las respuestas o entre un
refuerzo y el total de los refuerzos:

\[
\frac {R_1} {R_2} = \frac {r_1} {r_2}
\]

Como un segundo paso, Baum propuso que la razón de refuerzo es
transformada por el agente, como una función de potencia con dos
parámetros, similar a la propuesta por S. S. Stevens (AÑO) en la
psicofísica sensorial.

\[
\frac{R_1}{R_2} = \alpha \left( \frac{r_1}{ r_2} \right)^\beta
\]

Donde el parámetro \(\beta\) representa que tan \emph{sensible} es la
razón de respuesta a los cambios en la razón de refuerzos y el parámetro
\(\alpha\) representa el *sesgo** en la preferencia por una alternativa
sobre otra.

Una forma, visualmente más clara de ver la ecuación anterior, es su
transformación logarítmica:

\[
\log \frac{B_1}{B_2} = \beta \log\frac{r_1}{r_2} + \log \alpha
\]

En la figura x podemos ver su comportamiento y en el simulador uds.
pueden jugar con diferentes valores de los parámetros de sesgo y
sensibilidad. Podemos ver que bajo la transformación logarítmica, la
ecuación de potencia se convierte en una familia de líneas rectas en las
que \(\beta\) es la pendiente de la función y \(\alpha\) es su
intercepto. Cuando la sensibilidad es igual a uno y no hay sesgo (es
decir, \(\alpha\) = 0), la ecuación es la ley de igualación y la línea
recta parte del origen. Los valores de \(\alpha\) positivos o negativos
generan líneas rectas paralelas a la recta de igualación. Los valores
del parámetro de sensibilidad \(\beta\) menores a uno representan
sub-igualación y los valores mayores a uno representan sobre-igualación.
En el primer caso, la importancia de la diferencia entre los refuerzos
se empequeñece psicológicamente y en el segundo caso la misma diferencia
se agranda.

imagen

En el marco de referencia de los modelos de elección basados en el valor
de las consecuencias, la ecuación generalizada de igualación resulta ser
una instancia de una regla de respuesta en la que la probabilidad de
cada respuesta es una función de la diferencia entre reforzadores:

\[
P(a_1)= F(\lambda(Qa_1 -Qa_2))
\]

En nuestro caso, el parámetro \(\beta\) es \(\lambda\) y la función
\(F\) es una función logística aplicada a la diferencia de los
logaritmos de los dos refuerzos:

\[
\log \frac{R_1}{R_2} = \beta (\log{r_1} - \log{r_2}) + \log \alpha
\]

La ecuación generalizada de igualación puede emplearse para distintos
usos: ya sea evaluar la preferencia entre diferentes refuerzos
(representada por el valor del parámetro alfa); investigar bajo qué
condiciones se obtiene igualación perfecta, es decir, cuando
\(\alpha = 1\) y \(\beta = 1\) o finalmente, establecer la forma en la
que las diferencias en refuerzo son transformadas en distintas
distribuciones del comportamiento. El parámetro \(\beta\) puede
interpretarse en al menos dos formas, primero como una propiedad del
sistema: de la misma forma que el exponente para las funciones
psicofísicas varía dependiendo de la dimensión sensorial, la
sensibilidad beta podría variar dependiendo del tipo de refuerzo. En
segundo lugar, beta también puede interpretarse como el resultado de un
conjunto de manipulaciones experimentales y restricciones perceptuales
del sistema que imponen límites en la discriminabilidad de las
diferencias de refuerzos. Para dar un ejemplo de estas dos
interpretaciones plausibles ante un único fenómeno de discriminación:
consideremos la diferencia en el refuerzo que generan dos sabores
placenteros distintos, esta diferencia es mucho más fácil de distinguir
que aquella los refuerzos que otorgan dos sonidos melódicos distintos.
Esta diferencia en la facilidad o dificultad para discriminar entre dos
estímulos podría deberse a una propiedad intrínseca de los sabores como
fenómeno respecto a los sonidos (interpretación 1), aunque también
podría deberse a que nuestro sistema sensorial (el sistema sensorial
humano) tiene una mayor facilidad para distinguir refuerzos por vía de
la modalidad gustativa que por vía de la modalidad auditiva
(interpretación 2). La segunda interpretación sugiere que esta
dificultad discriminativa podría no presentarse en otras especies con
sistemas sensoriales distintos, por ejemplo, en los murciélagos, que
poseen sistemas auditivos más agudos que nosotros.

\subsection{Igualación como un Mecanismo
Adaptable}\label{igualaciuxf3n-como-un-mecanismo-adaptable}

Una respuesta común al principio de igualación es considerarlo como una
instancia de un mecanismo de adaptación, seleccionado para maximizar el
total de refuerzos disponibles. Para atender esta posibilidad, es
necesario conocer la función que relaciona, por un lado, a la suma del
total de los refuerzos, y por el otro, a las distribuciones relativas de
respuesta. Con ello, se puede evaluar si el patrón de respuesta de
igualación corresponde al máximo de la función.

\[
r_{total}= f\left(\frac{R_1}{R_1 + R_2}\right)
\]

Dado que en programas de IV los refuerzos no se cancelan hasta que se
obtienen, al animal le conviene seguir visitando ambas opciones y de esa
forma obtener todos los reforzadores posibles. Sin embargo, el número de
posibles distribuciones de respuestas que garantizarían obtener todos
los refuerzos es enorme. El organismo podría hacerlo simplemente
alternando constantemente cada respuesta (0.5) o pasando casi todo el
tiempo en una de las opciones con una ocasional visita a la opción no
atendida. Lo sorprendente es que dentro de ese amplio rango de posibles
tasas relativas de respuesta que producen maximización, lo que se
observa empíricamente es la proporción de respuesta a cada opción que
iguala la tasa de refuerzo relativa. La importancia de la igualación es
que representa la solución observada, en equilibrio, a la multiplicidad
de formas de maximizar el refuerzo total en programas concurrentes de
intervalo variable.

\subsection{¿Es Maximización el Mecanismo que Subyace a
Igualación?}\label{es-maximizaciuxf3n-el-mecanismo-que-subyace-a-igualaciuxf3n}

Una pregunta muy diferente a la de si la igualación es un comportamiento
adaptable es la de si, bajo condiciones de equilibrio, la maximización
de la tasa de reforzamiento global es el ``mecanismo'' que guía el
comportamiento del organismo y el cual subyace al patrón de respuestas
observado en igualación.~En otras notas veremos modelos en los que se
maximizan diferentes variables, pero en estas nos concentramos en la
maximización del número de refuerzos totales. Para comprender la
pregunta, es necesario considerar que los fenómenos de maximización e
igualación implican que los algoritmos que los organismos computan son
diferentes para cada modelo. De acuerdo a esta versión de la
maximización como mecanismo subyacente a la igualación, el algoritmo no
distingue entre las dos respuestas disponibles y el refuerzo asociado
con cada una de ellas: en lugar de ello, este solo computa y actualiza
dos variables, la suma de refuerzos y la tasa relativa de respuestas.
Noten que, debido a esto último, este modelo de acción no es una
instancia de un modelo de elección basado en el valor de las respuestas
individuales. En cambio, este modelo asume que los organismos cuentan
con solo dos contadores, uno para la tasa relativa de respuestas y otro
para la suma de los reforzadores obtenidos por las dos respuestas, sin
distinguir entre su origen. Un reloj acumula el tiempo total \emph{T},
durante el cual las dos respuestas se encuentran disponibles. El
resultado del contador del total de refuerzos se divide entre el tiempo
T. En estos casos, el organismo busca acceder al mayor número de
refuerzos por unidad de tiempo. Este número de \{r/T\} representa la
\emph{ganancia} asociada con cada distribución posible de respuestas.
Además, dentro de un proceso de ascenso de colina, como el visto en las
notas x, la tasa relativa de respuesta se mueve en la dirección de una
mayor tasa global de refuerzo hasta alcanzar un máximo, el cual puede
ser local.

\subsection{Igualación y Rentabilidad de las
Respuestas}\label{igualaciuxf3n-y-rentabilidad-de-las-respuestas}

Otra explicación que puede dar cuenta del patrón de respuesta de
igualación es que los organismos buscan igualar la rentabilidad de sus
respuestas o tiempos. La rentabilidad es el número de refuerzos que se
obtienen por tiempo o respuestas invertidos en una opción. Cuando
ustedes deciden entre planes de ahorro bancario, la primera pregunta que
hacen es cuál es la tasa de interés anual, lo que les permite saber
cuánto ganarán anualmente por cada \$1,000 pesos depositados en su
cuenta. Igualación sugiere que esto es exactamente lo que hacen los
agentes con la asignación de sus respuestas y tiempos: específicamente,
igualación plantea que la regla que siguen los agentes es distribuir sus
respuestas y tiempos de tal forma que, en equilibrio, las dos opciones
tengan la misma rentabilidad. En concreto, bajo este modelo
computacional, el organismo en esencia registra la tasa de refuerzo
asociada con cada opción de respuesta disponible y luego distribuye sus
respuestas proporcionalmente en cada opción para igualar las
rentabilidades. Igualación para respuestas y tiempos también puede
expresarse en forma de razones, esto es, en lugar de hablar de una
frecuencia relativa de 6 de 8 (0.75) refuerzos para una respuesta,
hablamos de una razón de 6 a 2 (3) refuerzos para esa respuesta:

\[
\frac {T_1} {T_2} = \frac {r_1} {r_2}
\]

\[
\frac {R_1} {R_2} = \frac {r_1} {r_2}
\]

Reacomodando términos:

\[
\frac {r_1}{T_1} = \frac {r_2} {T_2} \text{ y } \frac {r_1}{R_1} = \frac {r_2} {R_2}
\]

Esta nueva forma de expresar la ley de igualación refleja que lo que se
iguala es la rentabilidad de las distintas opciones de respuesta y/o
tiempo. De este modo, lo que los organismos igualan son lo que podemos
llamar como \emph{tasas locales de refuerzo}. Así, cuando un organismo
sigue la ley de igualación, este experimentará tasas iguales de
reforzamiento local en todas las opciones disponibles. Es decir, si el
organismo recibe un reforzador por cada 30 respuestas a la Opción A,
este ajustará su número de respuestas a la Opción B para igualar la tasa
de reforzamiento local de un reforzador por cada 30 respuestas. De
manera similar, si el organismo recibe un reforzador por cada 30
segundos dedicados a la Opción A, este ajustará sus respuestas para
recibir un reforzador por cada 30 segundos dedicados a la Opción B.

Es importante recalcar que para igualar las tasas de refuerzo local (la
tasa de refuerzo por respuesta o por unidad de tiempo) el número de
respuestas emitido por el organismo a las distintas opciones puede
variar sustancialmente.

Por ejemplo:

-Si la Opción A brinda 1 reforzador por cada 30 respuestas y la Opción B
brinda 1 reforzador por cada 60 respuestas, igualar las tasas de
refuerzo locales requeriría que el organismo respondiera dos veces más a
la Opción B que a la Opción A.

-Si distintos programas IV concurrentes se encuentran operando (por
ejemplo, IV 30 s para la Opción A e IV 60 s para la Opción B), igualar
las tasas de refuerzo local implicaría asignar distintas cantidades de
tiempo y respuestas a cada opción.

El organismo esencialmente ajusta su comportamiento para obtener un
``mismo rendimiento'' sobre el tiempo/energía que invierte en todas las
alternativas, lo que frecuentemente resulta en una distribución bastante
desigual de respuestas.

También, vale la pena notar que igualdad en las tasas de reforzamiento
locales (rentabilidad) ocurre naturalmente a través de la distribución
del comportamiento del organismo, independientemente de los diferentes
programas de intervalo variable programados para cada opción.

Computacionalmente, considerar que el algoritmo de igualación opera
dentro de programas concurrentes de dos respuestas implica que los
agentes tienen cuatro contadores, dos para respuestas y dos para
reforzadores, y adicionalmente, disponen de dos relojes que se echan a
andar cuando cambian a una de las opciones y que se detienen cuando
regresan a la opción visitada anteriormente. Estos relojes se usan para
computar las tasas locales de refuerzo.

Aqui falta el ejemplo numerico.

\subsection{Maximización vs
Rentabilidad}\label{maximizaciuxf3n-vs-rentabilidad}

¿Es posible distinguir entre estas dos interpretaciones de la elección
en programas concurrentes? Consideren el siguiente escenario. Una
estudiante es la única heredera de dos tías de edad avanzada. El monto
de la herencia que le deja una de ellas depende del número de visitas
que la sobrina le haga; por otra parte, la cantidad que le deja la otra
tía tiene un tope máximo y solo depende de que ella la visite
ocasionalmente. El escenario de las tías ilustra un programa
concurrente, con una de las opciones reforzada con un programa de
intervalo variable y la otra con un programa de razón variable. En este
escenario, la estrategia óptima de la sobrina es asignar la mayor parte
de sus visitas a la tía más demandante, la que ejemplifica un programa
de razón, y visitar ocasionalmente a la tía que ejemplifica el programa
de intervalo.

Herrnstein y Heyman (), llevaron al laboratorio el escenario recién
descrito. Expusieron a las palomas a programas concurrentes razón
variable - intervalo variable. En este arreglo, una de las respuestas es
reforzada de acuerdo a una regla temporal (programa de intervalo
variable), mientras que el refuerzo para la otra respuesta depende del
número de ellas (programa de razón variable).

Para entender la lógica del experimento, se debe tener presente que en
programas de razón variable, la rentabilidad de la respuesta es el
número de respuestas necesario para obtener un refuerzo. Por ejemplo, la
rentabilidad de un programa de razón variable 30 para una de las
opciones es un reforzador por cada treinta respuestas invertidas (1/30).
De acuerdo a la ley de igualación, en un programa concurrente RV30 -
IVx, el número de respuestas por refuerzo dentro del programa de
intervalo debe ser también de 30. Sin embargo, de acuerdo a una regla de
maximización global, la distribución óptima sería responder
mayoritariamente en la opción reforzada con el programa de razón y
visitar ocasionalmente la opción reforzada de acuerdo al programa de
intervalo.

La siguiente figura muestra los resultados obtenidos por Heyman y
Herrnstein. Puede verse que la distribución de respuestas de las palomas
iguala la distribución de refuerzos, con un sesgo en favor del programa
de razón. El patrón de igualación obtenido en este programa favorece al
algoritmo de rentabilidad por encima del de maximización. Experimentos
más recientes confirman este resultado y resaltan la utilidad de separar
el sesgo por una opción (en este caso la opción RV) de la sensibilidad
de los organismos hacia las diferencias en refuerzo de los dos
programas.

\bookmarksetup{startatroot}

\chapter{Comportamiento de Elección: Maximización
Local}\label{comportamiento-de-elecciuxf3n-maximizaciuxf3n-local}

En el capítulo anterior, vimos que estudiando una medida agregada de
respuestas y refuerzos, en equilibrio, la tasa relativa de respuestas
tiende a igualar a la tasa de refuerzo que produce cada opción.
Contemplamos dos posibles explicaciones computacionales para este
fenómeno: la igualación de la probabilidad de refuerzo para las opciones
de respuesta (igualación de la rentabilidad de las respuestas) y la
maximización de la tasa global de refuerzo.

Una alternativa a estos modelos molares son los modelos de
\emph{maximización local}, los que asumen que igualación es el resultado
de que en cada oportunidad de respuesta, los organismos eligen aquella
respuesta asociada con el valor más alto de alguna variable local. Vamos
a revisar tres miembros de esta familia de modelos, que se distinguen
entre ellos por la variable local que cada uno maximiza:

\begin{itemize}
\tightlist
\item
  \emph{Maximización Momentánea}: Modelos de elección basados en las
  probabilidades instantáneas de refuerzo de cada alternativa.
\item
  \emph{Mejoramiento}: Modelos de elección basados en las tasas locales
  de refuerzo asociadas con cada alternativa.
\item
  \emph{Valor Q de la respuesta}.
\end{itemize}

\section{Maximización Momentánea}\label{maximizaciuxf3n-momentuxe1nea}

Este modelo asume que los organismos computan las probabilidades locales
de refuerzo asociadas con cada respuesta. Consideren un programa
concurrente IV - IV. En esos programas, la probabilidad de un refuerzo
para una respuesta en cada tecla es una función del tiempo transcurrido
desde la última visita a una de ellas. Conforme incrementa el tiempo de
estancia en una opción, incrementa la probabilidad de un refuerzo para
la respuesta en la opción alterna. Este modelo fue propuesto por Shimp
en 1969, quien argumentó que el cambio de una opción a otra era
controlado por los cambios en la probabilidad de refuerzo asociados con
el tiempo transcurrido en cada opción. Para él, la igualación global era
el resultado de una regla de respuesta seguida por el organismo en la
cual este respondía a la tecla que tuviera la mayor probabilidad de
refuerzo al momento de la elección. Simulando este modelo, Shimp
encontró que a nivel global este algoritmo resulta efectivamente en
tasas de respuesta iguales a las tasas de refuerzo: reproduciendo así el
patrón de igualación.

\subsection{Evaluación experimental del modelo de maximización
momentánea}\label{evaluaciuxf3n-experimental-del-modelo-de-maximizaciuxf3n-momentuxe1nea}

\subsubsection{1. Programas de refuerzo concurrente IV - IV de ensayos
discretos}\label{programas-de-refuerzo-concurrente-iv---iv-de-ensayos-discretos}

Empíricamente, el modelo de maximización instantánea predice
regularidades en la estructura de los cambios del organismo de una
opción a otra después de ciertas secuencias de respuesta. Si se computan
las probabilidades de refuerzo de cada una de las opciones de respuesta,
es posible determinar cuál respuesta debe seguir después de una
secuencia de \emph{n} respuestas consecutivas en una misma opción. La
forma más accesible de estudiar esta predicción es empleando
procedimientos de programas de refuerzo concurrentes IV - IV, pero de
ensayos discretos. En estos programas, al animal se le presenta la
oportunidad de elegir entre dos alternativas con una sola respuesta. Las
dos opciones se presentan durante un breve periodo de tiempo. Después de
una respuesta a alguna de las dos alternativas, inicia un breve
intervalo entre ensayos sin opción de respuesta: al final de este
intervalo, al organismo se le presenta una vez más una breve oportunidad
para elegir entre una de las dos respuestas. Lo que es importante tener
presente frente a estos protocolos es que los programas de refuerzo
siguen corriendo durante los intervalos entre ensayos como si no hubiese
discontinuidades en el tiempo. Al igual que en los programas IV - IV que
hemos revisado, una vez que un refuerzo está disponible para una de las
opciones, el reloj del programa IV se detiene y el refuerzo se guarda
hasta que es recogido por el animal. Con este protocolo, mientras mayor
es el número de elecciones repetidas por una de las opciones, mayor es
la probabilidad de que un refuerzo esté esperando en la opción alterna.

En experimentos con programas de ensayos discretos, se mide la
probabilidad de cambiar de alternativa después de varias secuencias de
respuesta a una misma tecla. Mientras mayor sea el número de respuestas
seguidas a una misma tecla, mayor será la probabilidad de refuerzo
asociada con la respuesta alterna. De acuerdo al modelo de maximización
instantánea, debería observarse que la probabilidad de cambiar de
alternativa es una función del número de respuestas que se hayan dado a
la misma tecla. Por ejemplo, si el programa es uno concurrente de tecla
verde con IV 1' - tecla roja con IV 3', el modelo predice la siguiente
secuencia de tres respuestas: Verde-Verde-Roja. Por los valores de los
programas de intervalo, después de cada dos respuestas seguidas a la
opción verde resulta más probable que se otorgue un refuerzo a la opción
roja.

En el experimento más citado con este procedimiento, Nevin () no
encontró evidencia en favor del modelo de maximización instantánea. Ver
Figuras.

imagen

Obsérvese en la figura del panel izquierdo que la probabilidad de
refuerzo de la Opción Roja aumenta con la acumulación de elecciones de
la Opción Verde por parte del organismo. Al mismo tiempo, en la figura
del panel derecho, nótese que la probabilidad de cambiar a la Opción
Roja no aumenta con la acumulación de elecciones de la Opción Verde por
parte del organismo. Nota: la probabilidad de cambiar de opción se
calculó con base al número de oportunidades para cambiar de opción en
cada secuencia (run length). Así, estos resultados rompen con el patron
de elecciones esperado según el modelo de maximización instantánea.

En un análisis posterior de su experimento original y de los datos de un
experimento de Silberberg (), Nevin () encontró que la perseverancia en
las opciones de respuesta (independientemente de su probabilidad de
refuerzo) era el patrón más frecuentemente observado en ambos
experimentos. Los animales tendían a quedarse en la tecla a la que
habían respondido anteriormente y no a cambiar como una función del
número de respuestas a esa tecla Ver fig.

imagen

\subsubsection{2. Programas de refuerzo concurrente RV - IV de ensayos
discretos}\label{programas-de-refuerzo-concurrente-rv---iv-de-ensayos-discretos}

Un protocolo adicional para evaluar el modelo de maximización local es
un programa concurrente de ensayos discretos RV - IV. En estos
programas, la probabilidad de refuerzo para las respuestas asociadas a
la opción RV es constante, mientras que la probabilidad de refuerzo para
la respuesta asociada al programa IV cambia como una función de la
última respuesta a esa opción. Una estrategia consistente con el modelo
de maximización instantánea consiste en responder a la opción asociada
con el programa RV inmediatamente después de recibir un refuerzo en la
opción asociada con el programa IV. Este es el momento con menor
probabilidad de refuerzo para la opción IV. De igual forma, la
probabilidad de un cambio de la opción RV hacia la opción IV debe
incrementar como una función del número de respuestas que se han dado a
la opción RV.

La siguiente figura x presenta los resultados obtenidos por Williams ()
usando el protocolo anterior. En primer lugar, Williams encontró que los
animales igualaban la frecuencia relativa de respuestas a la frecuencia
relativa de refuerzos, pero de manera aún más importante para estas
notas: no encontró evidencia de que la respuesta fuese controlada por la
probabilidad instantánea de refuerzo. El panel izquierdo de la figura
muestra la probabilidad de refuerzo como una función del número de
ensayos desde la última respuesta a la opción IV. Puede verse que la
probabilidad de refuerzo para un cambio a la tecla IV es creciente;
consecuentemente, la probabilidad de una respuesta a la tecla IV también
debería incrementar como una función del número de ensayos desde la
última respuesta a la opción IV. Sin embargo, el panel derecho de la
figura muestra que la probabilidad real u observada de una respuesta al
IV es constante o decreciente.

imagen

\subsection{Conclusiones acerca del modelo de maximización
Instantánea}\label{conclusiones-acerca-del-modelo-de-maximizaciuxf3n-instantuxe1nea}

De los experimentos con el protocolo de ensayos discretos recién
descritos y muchos otros que no presentamos se pueden alcanzar las
siguientes conclusiones:

\begin{itemize}
\tightlist
\item
  Consistente con la ley del efecto, la probabilidad de que se repita
  una opción de respuesta incrementa si esta es seguida por un refuerzo.
\item
  También se observa un \emph{efecto de perseverancia}: es más probable
  que se repita la respuesta a una opción que ya fue elegida en el
  pasado, independientemente de si esta respuesta es reforzada o no.
\item
  Es importante agregar que la consistencia de los resultados descritos
  también depende de otras variables como el intervalo entre opciones de
  elección, el cual afecta la memoria del organismo sobre sus elecciones
  previas.
\end{itemize}

\section{Modelo de Mejoramiento}\label{modelo-de-mejoramiento}

El modelo de mejoramiento de Herrnstein y Vaughan () describe la
dinámica de elección en situaciones de elección recurrentes, momento a
momento. Como otros modelos de elección basados en el valor de las
opciones, el modelo de mejoramiento cuenta con dos componentes:

\begin{itemize}
\item
  El primer componente es la especificación de la variable de decisión a
  partir de la cual el organismo elige (es decir, la variable que guía
  la elección de los organismos; ejemplos de estas variables son la
  probabilidad de refuerzo, el tiempo transcurrido en una opción o las
  tasas locales de refuerzo, entre otras opciones).
\item
  El segundo componente es \emph{la regla de respuesta}: esto es, con
  qué criterio elige el organismo.
\end{itemize}

Para el modelo de mejoramiento, la variable de decisión son las tasas
locales de refuerzo, a las cuales previamente llamamos como la
rentabilidad de las opciones. Esta variable se computa dividiendo el
número de refuerzos para cada opción entre el tiempo asignado a cada una
de ellas:

\[\frac {r_i}{T_i}\]

La rentabilidad puede computarse también para respuestas:

\[\frac {r_i}{R_i}\]

Asumiendo un tiempo total fijo, de acuerdo al modelo de mejoramiento,
cada incremento en una unidad de tiempo asignada a una opción tiene como
consecuencia la actualización de las dos tasas de refuerzo locales. Al
incrementar el tiempo asignado \(t_i\) a una de las opciones, la tasa
local de refuerzo de esa opción disminuye (dado que el denominador de la
rentabilidad de esa opción crece); simultáneamente, este cambio también
reduce el tiempo \(t_2\) asignado a la otra opción (dado que el tiempo
total es fijo), lo cual incrementa la tasa de refuerzo local asociada
con la segunda opción (puesto que el denominador de esta segunda
rentabilidad se achica).

El segundo componente del modelo de mejoramiento, la regla de elección,
es una variante de maximización que consiste en seleccionar la opción
con la mejor tasa de refuerzo local dentro de cada oportunidad.

Este modelo describe la elección dentro de un programa concurrente que
se construye como un sistema dinámico y de retroalimentación. Es decir,
bajo este arreglo, la opción de respuesta que tiene la mayor tasa local
de refuerzo va cambiando como una función de la asignación de tiempos y
respuestas por parte del organismo. Es por ello que este modelo puede
dar cuenta de la dinámica global de acercamiento del organismo al patrón
de igualación observado en programas concurrentes IV - IV. Para entender
al modelo, tengan presente que al elegir la opción con la mejor tasa de
refuerzo local, el organismo está incrementando el tiempo de estancia en
esa opción: lo cual aumenta la base temporal \emph{t} para computar la
tasa de refuerzo local de esa opción. Simultáneamente, y dado el tiempo
total fijo, este comportamiento reduce el tiempo de estancia en la
segunda opción, lo cual incrementa la tasa local de refuerzo de esa
segunda opción. De esta forma, cada aumento de una unidad de tiempo de
estancia en la mejor opción por parte del organismo, tiene dos
consecuencias: reduce la tasa de refuerzo local de esa opción e
incrementa la tasa de refuerzo local de la otra opción. La transición
del organismo de una opción a la otra ocurre cuando la dirección de la
diferencia entre las dos tasas cambia de signo y, básicamente, la opción
alterna se vuelve mejor que la opción actual. Pueden observar que en
estos programas, el modelo de mejoramiento es uno de retroalimentación y
corresponde a un algoritmo cuyo blanco es la reducción de la diferencia
entre las dos tasas de refuerzo locales; el seguimiento de este
algoritmo a la larga (o en equilibrio) resulta en la igualación global
de las tasas de respuesta a las tasas de refuerzo para ambas opciones
(el patrón de igualación). Consideren como ejemplo un programa
concurrente IV 1 min - IV 2 min. Supongan que al inicio de la sesión de
una hora, el organismo asigna la mitad de su tiempo a cada alternativa.
Supuesto importante: Vamos a asumir que el animal responde a una tasa
moderada que garantiza que en ambos programas de IV el organismo
obtendrá el máximo número de reforzadores posibles (para 1 hora),
independientemente del tiempo que efectivamente dedique a cada opción
durante la sesión. Esto ocurre porque en los programas de intervalo
variable, los reforzadores se ``acumulan'' durante el tiempo que el
animal no está respondiendo a esa opción. Por lo tanto, aunque el animal
dedique menos de 1 hora a una opción, si este responde de manera
suficientemente rápida, efectivamente podría obtener todos los
reforzadores correspondientes a 1 hora pero en una menor cantidad de
tiempo. Dado este supuesto, las tasas de refuerzo locales para las dos
opciones se calculan como el número máximo posible de refuerzos en una
hora dividido por la proporción de tiempo realmente asignada a cada
opción. Así, la tasa local de refuerzo para la tecla IV 1' sería de 60
reforzadores máximos / 0.5 hr = 120 refuerzos por hora. Una vez más,
esto significa que aunque el animal solo dedica media hora a esta
opción, este puede obtener los 60 reforzadores disponibles durante la
sesión completa gracias a la acumulación propia de los programas IV.
Nótese que si el organismo le asigna toda la hora a responder a este
programa, su tasa local de refuerzo será de 60/1 hr = 60 refuerzos por
hora. Al reducir el tiempo asignado a esa opción a media hora (0.5 hr),
la rentabilidad de esa opción aumenta a 120 refuerzos por hora, aunque
el número absoluto de reforzadores (60) permanece constante. Para el IV
2', la tasa local de refuerzo es de 30 reforzadores máximos / 0.5 hr =
60 refuerzos por hora. Dados estos resultados y de acuerdo al modelo de
mejoramiento, el animal debería escoger asignar más tiempo a la opción
con el programa IV 1' que a la opción con el programa IV 2', ya que la
primera presenta una mayor tasa de refuerzo local (120 vs 60). Supongan
ahora que como resultado de asignar más tiempo a la opción IV 1', el
organismo termina dedicando el 90\% de su tiempo a esta opción. Ahora,
para la opción asociada con el IV 1 min, la tasa de refuerzo local sería
de 60/0.9 hr = 66.7 refuerzos por hora. En el IV 2 min, la tasa de
refuerzo local sería de 30/0.1 hr = 300 refuerzos por hora. Recordamos
que los programas en sí no han cambiado: lo único que está modificando
la rentabilidad percibida de ambas opciones es la manera en la que el
organismo distribuye su tiempo entre ellas. Pueden observar que al
incrementar el tiempo asignado al IV 1', paradójicamente se incrementó
la tasa local de refuerzo asociada con la opción IV 2' (de 60 a 300
refuerzos por hora) y por lo tanto, si el organismo sigue la regla de
mejoramiento, su subsecuente elección debe ser la opción del IV 2'. Bajo
este modelo, el equilibrio se alcanza cuando la proporción de tiempo
asignado a las dos opciones por parte del organismo es de 2/3 para IV 1'
(60/0.666 90) y 1/3 para IV 2' (30/0.333 90). Lo anterior es
aproximadamente igual a ``90 refuerzos por hora'' en ambas alternativas.
Así, en el punto de equilibrio, ambas alternativas ofrecen exactamente
la misma tasa de refuerzo local (90 por hora), lo que explica por qué el
organismo deja de cambiar su distribución de tiempo entre las opciones
tras alcanzar este punto. Este resultado es equivalente a la
\textbf{igualación} global de tasas relativas de respuesta a las tasas
relativas de refuerzo: 0.66 hr/(0.66 hr + 0.33 hr) 60 r/(60 r + 30 r) En
el correspondiente simulador, ustedes podrán ver la dinámica del sistema
para diferentes valores de programas concurrentes.

Un problema importante del modelo de mejoramiento es que deja sin
especificar la ventana temporal que los organismos requieren para
computar las tasas de refuerzo locales de las distintas opciones. ¿Estas
tasas se estiman hasta finalizar la duración de cada sesión
experimental? O bien aún, ¿seguirá el organismo algoritmos menos
evidentes? Por ejemplo, ¿reiniciar la computación de las tasas para las
dos opciones cada 10 minutos?¿O borrar la historia con las opciones de
respuesta experimentadas días atrás? Las respuestas a todas estas
preguntas no son evidentes. Y al mismo tiempo, estas tienen importantes
implicaciones para la aplicación del modelo a entornos volátiles, los
cuales frecuentemente cambian las condiciones de refuerzo para las
diferentes opciones a lo largo del tiempo. Este tema será abordado en
otras notas.

\section{\texorpdfstring{Modelo de Valor \emph{Q} de la
respuesta}{Modelo de Valor Q de la respuesta}}\label{modelo-de-valor-q-de-la-respuesta}

De los modelos que dan cuenta de la elección recurrente, el modelo de
valor \emph{Q} que vimos en las notas x es el más cercano a la ley del
efecto original planteada por Thorndike. La variable de decisión de este
modelo es el valor \emph{Q} adquirido por cada opción de respuesta. Este
valor representa la integración de la historia de reforzamiento de cada
opción que resulta de la regla del error de predicción. Por otra parte,
la regla de respuesta bajo este modelo consiste en la elección
probabilística de la respuesta con mayor valor \emph{Q} en cada
oportunidad. Presten atención a que en este modelo, cada respuesta
adquiere su valor exclusivamente como función de los refuerzos que
produce y es independiente del valor de las demás alternativas
presentes.

Por consecuente, una predicción importante de este modelo es que cuando
al organismo se le presenta la oportunidad de elegir entre dos
respuestas con diferentes valores \emph{Q}, su elección debe ser
independiente del contexto donde fue adquirido el valor \emph{Q} de cada
opción. Imaginen que en su ciudad hay tres cadenas de cafeterías (A, B y
C). Cerca de su casa y cerca de su trabajo, hay dos sucursales
disponibles (digamos que A y B están cerca de su casa; y B y C están
cerca de su trabajo). Una de las cadenas de cafetería, la B, tiene una
sucursal en los dos escenarios donde ustedes compran café (a la sucursal
cerca de su casa le llamaremos B' y a la sucursal cerca de su trabajo le
llamaremos B'\,'). Las sucursales de la cadena B tienen un logo que las
distingue. Digamos que sus menús son similares, por lo que las dos
sucursales B, aunque tienen sutiles diferencias en cuanto a su staff,
comparten a grandes rasgos el mismo programa de recompensas (B' =
B'\,'). En cambio, las cafeterías A y C son claramente distinguibles en
cuanto a calidad con relación a ambas cafeterías B. En resumen: A es
mucho mejor que B (A \textgreater{} B) y C es mucho peor que B (C
\textless{} B). Sin embargo, un domingo ustedes acuden a otra zona de la
ciudad y se dan cuenta de que las dos sucursales de las cafeterías B (B'
y B'\,') han sido reubicadas ahora en una misma y nueva zona de la
ciudad. Dado que los programa de recompensas de ambas opciones son
iguales (B' = B'\,'), ustedes deberían ser indiferentes entre ellas. Sin
embargo, es posible que los valores percibidos de B' y de B'\,' dependan
de cuál era el otro restaurante con el que ambas opciones competían
dentro de su contexto previo (A o C).\\
Una forma de evaluar experimentalmente esta predicción sobre la
relevancia del contexto previo de los refuerzos es presentar a los
animales con dos situaciones de elección diferentes: Programa
concurrente 1: El animal puede elegir entre: Opción A: reforzada con
intervalo variable IV(a) Opción B: reforzada con intervalo variable
IV(b) Programa concurrente 2: El animal puede elegir entre: Opción A:
reforzada con el mismo intervalo variable IV(a) que en la Situación 1
Opción B': reforzada con un intervalo variable IV(b') diferente al de la
Situación 1 Posteriormente, se evalúa la preferencia del animal entre
las opciones A de ambos programas concurrentes, las cuales comparten
exactamente el mismo valor de reforzamiento (Q). Bajo este protocolo,
cada par de opciones (cada componente o programa concurrente) se
encuentra vigente en períodos de tiempo separados dentro de una misma
sesión (formalmente a este arreglo se le conoce como programas múltiples
de refuerzo). Una vez alcanzado el equilibrio en cada programa
concurrente IV - IV: se presenta una nueva combinación de las opciones
comunes a ambos programas. De acuerdo a los modelos de refuerzo, de
mejoramiento y de valor Q, la elección en los periodos de prueba debe
reflejar los valores adquiridos por cada opción individual.

Belke () corrió una versión de este protocolo. Para uno de los
componentes de programas concurrentes (Componente A), los valores de las
opciones eran Tecla Blanca (IV 20) vs Tecla Roja (IV 40), mientras que
para el otro programa concurrente (el Componente B), las opciones eran
Tecla Verde con IV 40 vs una Tecla Amarilla con IV 80. Así, para el
Componente A, la opción del IV 40 ocurría en el contexto de una opción
IV 20 (la Tecla Roja) que era dos veces más rica. Mientras tanto, para
el segundo par de estímulos (el Componente B), la opción IV 40 (la Tecla
Verde) se presentaba en el contexto de otra opción reforzada con un IV
80 (la Tecla Amarilla), la cual era dos veces menos rica. De acuerdo al
modelo \emph{Q} que no considera el contexto de los reforzadores, las
dos opciones de respuesta reforzadas con el IV 40 (las Teclas Roja y
Verde) en ambos programas deben tener el mismo valor. Sin embargo, otra
posibilidad es que el valor adquirido por una opción de respuesta
dependa del contexto de los refuerzos proporcionados a las respuestas
alternativas presentes. En este caso, la Tecla Verde reforzada con el IV
40 dentro del contexto de otra opción reforzada con un IV pobre,
adquiere un valor más grande que la Tecla Roja, también reforzada con un
IV 40 pero que fue entrenada previamente en un contexto con una opción
reforzada con un IV más rico.

Contrario a la predicción de indiferencia arrojada por el modelo de
refuerzo \emph{Q}, Belke encontró que los animales sí preferían la
opción IV 40 entrenada en el contexto pobre, por encima de la misma
opción entrenada en el contexto rico. Esta evidencia sugiere que el
impacto de los reforzadores sobre el valor de una respuesta depende de
los reforzadores obtenidos por las otras opciones presentes en el pasado
(es decir, con qué otros estímulos ha convivido cada estímulo
previamente).

Los resultados anteriores son consistentes con la siguiente
interpretación más local de la ejecución en programas concurrentes. La
interpretación consiste en suponer que en estos programas lo que el
animal aprende es el tiempo que debe pasar en una opción antes de
cambiar a otra alternativa. Suponemos que estos tiempos se encuentran
relacionados linealmente a la tasa de refuerzo en la tecla alternativa:
así, mientras más pobre es el programa alterno, mayor será el tiempo que
el organismo pasará en una alternativa. De esa forma, en la prueba de
Belke, la preferencia por el ``IV 40 que fue entrenado con el IV 80''
respecto al ``mismo IV 40 que fue entrenado con el IV 20'' se explica
porque en el primer programa concurrente (Componente B), el organismo
aprendió que debía pasar más tiempo por visita en la opción IV 40,
mientras que en el segundo programa (Componente A), este aprendió que
debía pasar menos tiempo por visita en la opción IV 40.

\section{Reflexiones Finales Sobre los Modelos de Elección Basados en
Valor}\label{reflexiones-finales-sobre-los-modelos-de-elecciuxf3n-basados-en-valor}

\begin{itemize}
\tightlist
\item
  La igualación de tasas relativas de respuesta a tasas relativas de
  refuerzo es un fenómeno muy robusto.
\item
  Igualación es el resultado de un proceso estabilizador de
  retroalimentación, gobernado por las propiedades temporales de los
  programas de intervalo.
\item
  Igualación ilustra la importancia de entender a los programas de
  refuerzo como restricciones temporales o de respuesta sobre la
  distribución de comportamientos.
\item
  En un nivel molecular, la dinámica del movimiento hacia igualación
  ilustra la relevancia del principio de refuerzo, entendido como un
  algoritmo de ascenso de colina, como el que vimos en las notas x. Este
  algoritmo postula que el sistema compara, en cada oportunidad de
  respuesta, el valor de las variables de decisión asociadas con cada
  alternativa y elige aquella con el valor más alto. La distribución del
  comportamiento entre diferentes opciones (en equilibrio) es el
  resultado de diversos factores: primero, el seguimiento del algoritmo
  de ascenso de colina por parte del organismo; segundo, los ajustes que
  sufre dicho algoritmo al operar bajo las diferentes restricciones
  impuestas por los distintos programas de refuerzo; y tercero, la regla
  de aprendizaje de los valores de la variable de decisión. *La
  importancia del tiempo de estancia reforzado en cada opción sobre la
  elección de los organismos.
\item
  El avance en los modelos de elección requiere de la consideración de
  protocolos que capturen la incertidumbre y la volatilidad propia de
  los entornos naturales de los organismos, un tema que quedará para
  otra nota.
\end{itemize}

\bookmarksetup{startatroot}

\chapter*{Referencias}\label{referencias}
\addcontentsline{toc}{chapter}{Referencias}

\markboth{Referencias}{Referencias}


\backmatter

\end{document}
